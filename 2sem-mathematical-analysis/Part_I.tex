\chapter{Определенный интеграл}
\section{Основные понятия}
\begin{opred}
\mybold{Разбиением отрезка $[a,b]$} называется набор $\underset{0\leq k\leq n}{\{x_k\}}=\{x_0,x_1,\ldots,x_n\}$, где $x_0=a$, $x_n=b$, $x_0<x_1<\ldots<x_n$.
\end{opred}
\begin{opred}
\mybold{Диаметром,} или \mybold{мелкостью разбиения $\{x_k\}$} называется число $d=d(\{x_k\})=\underset{1\leq k\leq n}{max}\{\Delta x_k\}$, $\Delta x_k=x_k-x_{k-1}$.
\end{opred}
\begin{opred}
\mybold{Размеченным разбиением} называется разбиение, в котором зафиксированы точки $\underset{1\leq k\leq n}{\{\xi_k\}}$, где $\xi_k\in[x_{k-1},x_k]$.
\end{opred}
\begin{opred}
Разбиение $\{y_m\}$ называется \mybold{измельчением разбиения $\{x_k\}$}, если $\{x_k\}\subset\{y_m\}$.
\end{opred}
\begin{opred}
Разбиение $\{z_j\}$ называется \mybold{объединением разбиений $\{x_k\}$ и $\{y_m\}$}, если $\{z_j\}=\{x_k\}\cup\{y_m\}$.
\end{opred}
\begin{opred}
Пусть на $[a,b]$ задана $f(x)$. \mybold{Интегральной суммой для $f(x)$ на отрезке $[a,b]$, составленной по размеченному разбиению $(\{x_k\},\{\xi_k\})$} называется выражение вида $$\sigma_f=\sigma_f(\{x_k\},\{\xi_k\}):=\sum\limits^n_{k=1}f(\xi_k)\Delta x_k$$
\end{opred}
\begin{opred}
Число $A$ наывается \mybold{пределом интегральных сумм $f(x)$ на $[a,b]$ при $d\to0$}, где $d$ - мелкость разбиений, если $\forall\ \eps>0\ \exists\ \delta=\delta(\eps)>0$, что для любого размеченного разбиения $(\{x_k\},\{\xi_k\})$, мелкость которого $d<\delta$, выполнено неравенство: $|\sigma_f\rr-A|<\eps$.
$$A=\lim\limits_{d\to0}\sigma_f\rr$$
\end{opred}
\begin{opred}
\mybold{Определенным интегралом} $f(x)$ на отрезке $[a,b]$ называется предел интегральных сумм этой функции на этом отрезке при $d\to0$.
$$\int\limits^b_af(x)dx:=\lim\limits_{d\to0}\sum\limits^n_{k=1}f(\xi_k)\Delta x_k$$
\end{opred}
\begin{theor}
Если у $f(x)$ существует предел интегральных сумм, то этот предел -- единственный.
\end{theor}
\begin{proof}
Предположим противное. Пусть существует 2 предела: $A_1<A_2,\ A_2-A_1=\alpha>0$.
По определению предела, для $\forall\ \eps=\frac\alpha3\ \exists\ \delta_1,\delta_2$, что:
\begin{enumerate}
\item для $\forall\ \rr,\ d<\delta_1\colon|\sigma_f\rr-A_1|<\eps$
\item для $\forall\ \rr,\ d<\delta_2\colon|\sigma_f\rr-A_2|<\eps$
\end{enumerate}
Тогда для любого размеченного разбиения $\rr$, у которого $d\leq min(\delta_1,\delta_2)$, будет выполнено и 1), и 2) \then $\sigma_f\rr$ попадает одновременно в 2 непересекающихся интервала. Противоречие.
\end{proof}
\begin{theor}
Если существует $\int\limits_a^bf(x)dx$, то обязательно $f(x)$ ограничена на $[a,b]$.
\end{theor}
\begin{proof}
Предположим противное. Пусть $f(x)$ неограничена на $[a,b]$. Тогда для любого разбиения $\{x_k\}\ f(x)$ будет неограничена на хотя бы одном отрезке $\xko$ этого разбиения. Выберем последовательность разбиений $\underset{0\leq k\leq n}{\{x_k^m\}}$ с мелкостью $d_m=d_m(\{x_k^m\})<\frac1m$. В каждом из этих разбиений выберем отрезок $\xmko$, где $f(x)$ не ограничена. Теперь подберем разметку так, чтобы интегральные суммы $\sigma_f\rr$ были больше $m$. Выберем $\xi_k$ на всех отрезках $\xmk$, кроме $\xmko$, произвольным образом. А на $\xko$ выберем $\xi_{k_0}$ так, чтобы $|f(\xi_{k_0})|>\frac{m+|\sum\limits_{k\neq k_0}f(\xi_k)\Delta x_k)|}{\Delta x^m_{k_0}}$.
Тогда вспомним неравенство: $|a+b|\geq|a|-|b|$ ($|a|=|(a+b)-b|\leq|a+b|+|b|$ \then $|a|-|b|\leq|a+b|$, ЧТД).

$|\sigma_f\rr|=|f(\xi_k^m)\Delta x_{k_0}^m+\sum\limits_{k\neq k_0}f(\xi^m_k)\Delta x^m_k|\geq|f(\xi_k^m)\Delta x_{k_0}^m|-|\sum\limits_{k\neq k_0}f(\xi^m_k)\Delta x^m_k|>m$ \then $\underset{d<\frac1m}{\sigma_f}>m$ \then при $m\to\infty$ получим противоречие.
\end{proof}
\begin{opred}
Пусть $f(x)$ ограничена на $[a,b]$, и задано размеченное разбиение $\rr$ \then $f(x)$ ограничена на каждом отрезке $\xk$ \then  существует $\underset{x\in\xk}{sup}\{f(x)\}=M_k$, $\underset{x\in\xk}{inf}\{f(x)\}=m_k$. \mybold{Верхней (нижней) интегральной суммой (суммой Дарбу)} $f(x)$ по разбиению $\{x_k\}$ на $[a,b]$  называется выражение:

$\vs:=\sum\limits_{k=1}^nM_k\Delta x_k$

$\ns:=\sum\limits_{k=1}^nm_k\Delta x_k$
\end{opred}
\begin{theor}
6 свойств сумм Дарбу.
\begin{enumerate}
\item $\forall\ \rr \ns\leq\sigma_f\rr\leq\vs$
\item $\forall\ \eps>0\ \exists$ разметка $\{\xi_k\}$ данного разбиения $\{x_k\}$, что $\sigma_f\rr-\ns<\eps$, $\vs-\sigma_f\rr<\eps$
\item При ризмельчении разбиения $\ns$ не может уменьшиться, $\vs$ -- увеличиться.
\item При добавлении к разбиению $\{x_k\}$ $q$ новых точек $\vs$ может уменьшиться не более чем на $(M-m)qd$, $d$ -- мелкость $\{x_k\}$. Аналогично для $\ns$.
\item Пусть $\{x_k\},\ \{y_j\}$ -- 2 разбиения $[a,b]$. $\vs,\ \ns$ и $\vs',\ \ns'$ -- их суммы Дарбу. Тогда $\ns\leq\vs'$, $\vs\geq\ns'$.
\item В силу 5), $\exists\ sup\{\ns\}=\underline{I}$ -- нижний интеграл Дарбу, $\exists\ inf\{\vs\}=\overline{I}$ -- верхний интеграл Дарбу, причем $\ns\leq\underline{I}\ \leq\ \overline{I}\leq\vs$.
\end{enumerate}
\end{theor}
\begin{proof}
\begin{enumerate}
\item Для любого разбиения $\{x_k\}$ и любой разметки $\{\xi_k\}$ $m_k\leq f(\xi_k)\leq M_k$ \then $\ns=\sum\limits^n_{k=1}m_k\Delta x_k\leq\sum\limits^n_{k=1}f(\xi_k)\Delta x_k=\sigma_f\rr\leq\sum\limits^n_{k=1}M_k\Delta x_k=\vs$
\item По определению $sup$, $\forall\ \eps>0$ на каждом $\xk\ \exists\ \xi_k$, что $M_k-f(\xi_k)<\frac\eps{b-a},\ 1\leq k\leq n$ \then $\vs-\sum\limits^n_{k=1}f(\xi_k)\Delta x_k=\sum\limits^n_{k=1}(M_k-f(\xi_k))\Delta x_k<\frac\eps{b-a}\sum\limits^n_{k=1}\Delta x_k=\eps$. Аналогично для $\ns$.
\item Достаточно доказать, что $\vs$ не увеличивается, а $\ns$ не уменьшается, при добавлении к разбиению $\{x_k\}$ 1 новой точки.

Пусть новая точка $\eta$ добавлена между $x_{k_0-1}$ и $x_{k_0}$. Расмотрим суммы Дарбу.

$\vs=M_{k_0}\Delta x_{k_0}+\sum\limits_{k\neq k_0}M_k\Delta x_k$

$\vs'=M_{k_0}^1(\eta-x_{k_0-1})+M_{k_0}^2(x_{k_0}-\eta)+\sum\limits_{k\neq k_0}M_k\Delta x_k$

Сравним эти два выражения. Заметим, что $M_{k_0}\Delta x_{k_0}=M_{k_0}(\eta-x_{k_0-1})+M_{k_0}(x_{k_0}-\eta)$. Причём очевидно, что $M_{k_0}\geq\underset{x\in[x_{k_0-1},\eta]}{sup}\{f(x)\}=M_{k_0}^1$ и $M_{k_0}\geq\underset{x\in[\eta,x_{k_0}]}{sup}\{f(x)\}=M_{k_0}^2$ \then $\vs\geq\vs'$. Аналогично для $\ns$.
\item Докажем, что при добавлении 1 новой точки к разбиению $\{x_k\}\ \vs$ может уменьшиться не более, чем на $(M-n)d$, где $M=\underset{x\in[a,b]}{sup}\{f(x)\},\ m=\underset{x\in[a,b]}{inf}\{f(x)\},\ d$ -- мелкость разбиения $\{x_k\}$.

Аналогично доказательству 3), пусть добавлена новая точка $\eta$ между $x_{k_0-1}$ и $x_{k_0}$. Рассмотрим разность $\vs-\vs'$:

$\vs-\vs'=M_{k_0}\Delta x_{k_0}-(M^1_{k_0}(\eta-x_{k_0-1})+M^2_{k_0}(x_{k_0}-\eta))=(M_{k_0}-M^1_{k_0})(\eta-x_{k_0-1})+(M_{k_0}-M^2_{k_0})(x_{k_0}-\eta)\leq(M-m)((\eta-x_{k_0-1})+(x_{k_0}-\eta))=(M-m)\Delta x_{k_0}\leq(M-m)d$
\item Пусть даны 2 любых разбиения: $\{x_k\},\ \{y_j\}$; $\{z_m\}=\{x_k\}\cup\{y_j\}$. Пусть $\vs,\ns$ -- суммы Дарбу для $\{x_k\}$, $\vs',\ns'$ -- для $\{y_j\}$, $\vs'',\ns''$ -- для $\{z_m\}$.

$\ns\leq\ns''\leq\vs''\leq\vs'$, $\ns'\leq\ns''\leq\vs''\leq\vs$.
\item Докажем, что $\ns\leq\nis\ \leq\ \vis\leq\vs$.

Предположим противное. $\vis<\nis,\ \nis-\vis=\alpha>0$. По определению $sup,\ inf$ для $\frac\alpha3\ \exists\ \ns$, что $\nis-\frac\alpha3<\ns\leq\nis;\ \exists\ \vs$, что $\vis\leq\vs<\vis+\frac\alpha3$ \then $\vs<\vis+\frac\alpha3<\nis-\frac\alpha3<\ns$ \then противоречие.
\end{enumerate}
\end{proof}
\begin{opred}
$f(x)$ называется \mybold{интегрируемой по Риману на $[a,b]$}, если $\exists\ \int\limits^b_af(x)dx$. Также используется запись $f\in \mathbb{R}[a,b]$.
\end{opred}
\section{2 критерия интегрируемости функции по Риману}
\begin{theor}Критерий интегрируемости в терминах сумм Дарбу

Для того, чтобы $f(x)$, ограниченная на $[a,b]$, была интегрируема по Риману, необходимо и достаточно, чтобы $\forall\ \eps>0\ \exists\ \{x_k\}$, что $\vs_f-\ns_f<\eps$.
\end{theor}
\begin{proof}\mybold{Необходимость.}

Пусть $\exists\ I=\int\limits^b_af(x)dx=\underset{d\to0}{\lim}\ \sum\limits^n_{k=1}f(\xi_k)\Delta x_k$ \then по определению $\lim,\ \forall \eps>0\ \exists\ \delta=\delta(\frac\eps4)>0$, что для любого разбиения $\{x_k\}$ и любой его разметки $\{\xi_k\}$, если $d(\{x_k\})<\delta$ \then $|\sigma_f-I|<\frac\eps4$.

По 2 свойству из теоремы 3, $\exists\ \{\xi_k'\},\ \exists\ \{\xi_k''\}$, что при даном разбиении $\{x_k\}$

$\sigma_f'=\sum\limits^n_{k=1}f(\xi_k')\Delta x_k,\ \sigma_f''=\sum\limits^n_{k=1}f(\xi_k'')\Delta x_k$, удовлетворяющие неравенствам:

$\sigma_f'-\ns<\frac\eps4,\ \vs-\sigma_f''<\frac\eps4$.

При этом выполнены и неравенства:

$|\sigma_f'-I|<\frac\eps4,\ |\sigma_f''-I|<\frac\eps4$ \then $|\vs-\ns|\leq|\vs-\sigma_f''|+|\sigma''-I|+|I-\sigma'|+|\sigma'-\ns|<\frac\eps4+\frac\eps4+\frac\eps4+\frac\eps4=\eps$
\end{proof}
Для доказательства достаточности нам потребуется доказать следующее утверждение:
\begin{lemma}Основная лемма Дарбу.

$\nis=\underset{d\to0}{\lim}\{\ns\},\ \vis=\underset{d\to0}{\lim}\{\vs\}$
\end{lemma}
\begin{proof}
По определению $\vis=inf\{\vs\}$ \then $\forall\ \eps>0\ \exists$ разбиение $\{x_k\}$, что $\vs$ удовлетворяет неравенству:

$\vis\leq\vs<I+\frac\eps2$

Пусть в $\{x_k\}$ имеется $q+1$ точка. Рассмотрим теперь разбиение $\{y_j\}$ с мелкостью

$d(\{y_j\})<\frac\eps{2(M+m)(q-1)}=\delta(\eps)>0$

Рассмотрим разбиение $\{z_m\}=\{y_j\}\cup\{x_k\}$. В $\{z_m\}$, по сравнению с $\{y_j\}$, добавлено не более, чем $q-1$ точек. Обозначим верхние суммы Дарбу: $\vs'$ - для $\{y_j\}$, $\vs''$ - для $\{z_m\}$.

$\vs''\leq\vs$, $\vs''<\vs'$

$\vs'-\vs''\leq(M-m)(q-1)d<(M-m)(q-1)\frac\eps{2(M-m)(q-1)}=\frac\eps2$

$I\leq\vs''<I+\frac\eps2$ \then $\vis\leq\vs'<\vis+\eps$ \then $\vis=\underset{d\to0}{\lim}\ \vs$
\end{proof}
\begin{proof}\mybold{Достаточность.}

Дано: $\forall\ \eps>0\ \exists\ \{x_k\}$, что $\vs-\ns<\eps$ \then $\ns\leq\nis\leq\vis\leq\vs$ \then $\nis=\vis=I$

По основной лемме Дарбу $\underset{d\to0}{\lim}\ \ns=\underset{d\to0}{\lim}\ \vs=I$

По свойству 1 теоремы 3, $\ns\leq\sigma_f(\{x_k\},\{\xi_k\})\leq\vs$ \then $\exists\ \lim\ \sigma_f(\{x_k\},\{\xi_k\})=I$, т. е. $f\in\mathbb{R}[a,b]$
\end{proof}
\begin{theor}Критерий интегрируемости в теминах верхних и нижних интегралов Дарбу

$f\in\mathbb{R}[a,b]\ \Leftrightarrow\ \nis=\vis$
\end{theor}
\begin{proof}\mybold{Необходимость.}

Пусть $f\in\mathbb{R}[a,b]$ \then по теореме 4 $\forall\ \eps\ \exists\ \{x_k\}$, что $\vs-\ns<\eps\ \Rightarrow\ \nis=\vis$

\mybold{Достаточность.}

Пусть $\nis=\vis$. По основной лемме Дарбу $\underset{d\to0}{\lim}\ \ns=\underset{d\to0}{\lim}\ \vs=I$. По теореме 3 $\ns\leq\sigma_f\leq\vs$. По теореме о двух милиционерах $\underset{d\to0}{\lim}\ \sigma_f=I$
\end{proof}
\section{Классы интегрируемых функций}
\begin{theor}
Если $f(x)$ непрерывна на $[a,b]$, то она интегрируема.
\end{theor}
\begin{proof}
По теорема Кантора $f(x)$ равномерно непрерывна на $[a,b]$, т. е. $\forall\ \eps>0\ \exists\ \delta=\delta(\eps)>0,\ \forall\ x_1,\ x_2\in[a,b]\colon|x_1-x_2|<\delta\Rightarrow|f(x_1)-f(x_2)|<\eps$.

Рассмотрим разбиение $\{x_k\}$ на $[a,b]$ с мелкостью $d\leq\delta(\frac\eps{b-a}$. Тогда $\vs-\ns=\sum\limits^n_{k=1}(M_k-m_k)\Delta x_k$, где $M_k-m_k=\underset{x\in[x_{k-1},x_k]}{sup}\{f(x)\}-\underset{x\in[x_{k-1},x_k]}{inf}\{f(x)\}=\underset{x_1,x_2\in[x_{k-1},x_k]}{sup}\{\underset{\leq\frac\eps{b-a}}{\underbrace{f(x_1)-f(x_2)}}\}\leq\frac\eps{b-a}\ \Rightarrow\ \vs-\ns\leq\sum\limits^n_{k=1}(M_k-m_k)\Delta x_k\leq\frac\eps{b-a}*(b-a)=\eps$ \then по теореме 4 $f\in\mathbb{R}[a,b]$.
\end{proof}
\begin{theor}
Если ограниченная функция $f(x)$ монотонна на $[a,b]$, то она интегрируема.
\end{theor}
\begin{proof}
Пусть $f(x)$ не убывает на $[a,b]$. Для $\forall\ \eps>0$ рассмотрим любое разбиение ${x_k}$ на $[a,b]$ с мелкостью $d\leq\delta(\eps)=\frac\eps{f(b)-f(a)}$. Тогда $\vs-\ns=\sum\limits_{k=1}^n(M_k-m_k)\Delta x_k$.

Так как $f(x)$ не убывает, то $\vs-\ns=\sum\limits^n_{k=1}(M_k-m_k)\underset{\leq d=\delta(\eps)}{\underbrace{\Delta x_k}}\leq\frac\eps{f(b)-f(a)}\sum\limits^n_{k=1}(M_k-m_k)=\frac\eps{f(b)-f(a)}(M_n-m_1)^*=\frac\eps{f(b)-f(a)}(f(b)-f(a))=\eps$.

По 1 критерию интегрируемости $f(x)$ интегрируема.

${}^*\colon$ действительно, $M_k=m_{k+1}$
\end{proof}
\begin{opred}
Функция $f(x)$ называется \mybold{почти везде непрерывной на $[a,b]$}, если для $\forall\ \eps>0$ существует конечный набор интервалов суммарной длины $l<\eps$, покрывающих все точки разрыва $f(x)$.
\end{opred}
\begin{theor}
Если $f(x)$ почти везде непрерывна на $[a,b]$, то она интегрируема.
\end{theor}
\begin{proof}
Для $\forall\ \eps>0$ рассмотрим конечный набор интервалов $J_j=(c_j,d_j)$, сумма длин которых $l=\sum\limits^q_{j=1}|J_j|<\frac\eps{2(M-m)}$, который покрывает все точки разрыва $f(x)$.

Здесь $M=\underset{a\leq x\leq b}{sup}\{f(x)\},\ m=\underset{a\leq x\leq b}{inf}\{f(x)\}$. (можно считать, что $J_j$ не пересекаются)

Тогда $[a,b]\setminus\underset{q}{\overset{j=1}{\cup}}J_j=I$ - объедиение отрезков, $I=\underset{q+1}{\overset{i=1}{\cup}}I_i$.

Рассмотрим $f(x)$ на $I_i$. Она там непрерывна \then по теореме Кантора $f(x)$ равномерно непрерывна на $I_i$, т. е. для $\forall\ \eps>0\ \exists\ \delta=\delta(\frac\eps{2(b-a)})>0$, что для $\forall\ x',x''\in I_i,\ |x'-x''|<\delta_i\ \Rightarrow\ |f(x')-f(x'')|<\frac\eps{2(b-a)}$

Тогда для любого разбиения $\nu_i=\{\nu_{i_k}\}$ отрезка $I_i$ с мелкостью $d_i<\delta_i(\frac\eps{2(b-a)})$ для любых двух точек элементарного отрезка разбиения($[\nu_{i_{k-1}},\nu_{i_k}]$):

$x',x''\in[\nu_{i_{k-1}},\nu_{i_k}]\ \Rightarrow\ |f(x')-f(x'')|<\frac\eps{2(b-a)}$.

Переходя к $sup$, получим:

$$M_{i_k}(=\underset{x\in[\nu_{i_{k-1}},\nu_{i_k}]}{sup}f(x))-m_{i_k}(=\underset{x\in[\nu_{i_{k-1}},\nu_{i_k}]}{inf}f(x))=\underset{x',x''\in[\nu_{i_{k-1}},\nu_{i_k}]}{sup}|f(x')-f(x'')|\leq\frac\eps{2(b-a)}\eqno{(\#)}$$

Возьмем теперь $0<\delta\leq\underset{1\leq i\leq q+1}{min}\{\delta_i\}$ и рассмотрим на любом отрезке $I_i$ разбиение мелкостью $d_i<\delta$ \then для любого $i\colon1\leq \leq q+1$ выполняется $(\#)$

Объединим все эти разбиения. Получится некоторое разбиение $\{y_k\}$ отрезка $[a,b]$, в которое войдут замыкания выброшенных интервалов $J_j$. Пусть $\vs,\ns$ - суммы Дарбу этого разбиения. Тогда рассмотрим их разность$\colon\vs-\ns=$

$=\sum\limits^N_{k=1}(M_k-m_k)\Delta y_k=\sum\limits_{[y_{k-1},y_l]\subset\cup I_i}(M_k-m_k)\Delta y_k+\sum\limits_{[y_{k-1},y_k]\subset\underset{=\cup J_j}{\underbrace{[a,b]\setminus\cup I_i}}}(M_k-m_k)\Delta y_k\leq\frac\eps{2(b-a)}\underset{\leq(b-a)}{\underbrace{\sum\limits_{[y_{k-1},y_l]\subset\cup I_i}\Delta y_k}}+$

$+(M-m)\underset{<\frac\eps{2(M-m)}}{\underbrace{\sum\limits_{[y_{k-1},y_l]\subset\cup J_j}\Delta y_k}}<\frac\eps{2(b-a)}(b-a)+\frac\eps{2(M-m)}(M-m)=\eps$
\end{proof}
\begin{theor}
Верно также следующее утверждение (без доказательства):

Если $f(x)$ интегрируема на $[a,b]$, а $\phi(y)$ - непрерывна на $[m;M]$, то $\phi(f(x))\in\mathbb{R}[a,b]$

Следствие: если $f\in\mathbb{R}[a,b]$, то и $\cfrac1f$ - тоже. ($f(x)\neq0$ на $[a,b]$).
\end{theor}
\section{Основные свойства определенных интегралов}
\begin{theor}7 свойств определенных интегралов

Соглашение: будем считать, что $\int\limits^a_af(x)dx=0$, $\int\limits^a_bf(x)dx=-\int\limits^b_af(x)dx\ (a<b)$
\begin{enumerate}
\item Линейность: $\forall\ \alpha,\beta\in\R,\ f,g\in\R[a,b]\colon\int\limits^b_a(\alpha f(x)+\beta g(x))dx=\alpha\int\limits^b_af(x)+\beta\int\limits^b_ag(x)dx$
\item Интегрируемость произведения: если $f,g\in\R[a,b]$, то $fg\in\R[a,b]$
\item Аддитивность: если $f\in\R[a,b]$, то $f\in\R[c,d]\ \forall\ [c,d]\subset[a,b]$.

Кроме того, $\forall\ c\in(a,b)\ \int\limits^b_af(x)df=\int\limits^c_af(x)dx+\int\limits^b_cf(x)dx$
\item
\begin{enumerate}
\item Если $f\in\R[a,b]$ и $f(x)\geq0,\ a\leq x\leq b$ \then $\int\limits^b_af(x)df\geq0$
\item Если $f$ непрерывна и неотрицательна на $[a,b]$, и существует $c,\ a\leq c \leq b,\ f(c)>0$, то $\int\limits^b_af(x)dx>0$
\end{enumerate}
\item
\begin{enumerate}
\item Если $f,g\in\R[a,b],\ f(x)\leq g(x),\ a\leq x\leq b$, то $\int\limits^b_af(x)dx\leq\int\limits^b_ag(x)dx$
\item Если $f,g$ непрерывны на $[a,b],\ f(x)\leq g(x),\ \exists\ c\in[a,b]\colon f(x)<g(c)$, то $\int\limits^b_af(x)dx<\int\limits^b_ag(x)dx$
\end{enumerate}
\item Если $f(x)$ неотрицательна и непрерывна на $[a,b]$ и $\int\limits^b_af(x)dx=0$, то $f(x)\equiv0$ на $[a,b]$
\item Если $f\in\R[a,b]$, то $|f|\in\R[a,b]$. Кроме того, $|\int\limits^b_af(x)dx|\leq\int\limits^b_a|f(x)|dx$
\end{enumerate}
\end{theor}
\begin{proof}$\phantom{Хуй!}$
\begin{enumerate}
\item Рассмотрим интегральные суммы функций $f,g,\alpha f,\beta g$ по некоторому размеченному разбиению $\rr$:
$$
\sigma_{\alpha f+\beta g}=\sum\limits^n_{k=1}(\alpha f(\xi_k)+\beta g(\xi_k))\Delta x_k=\alpha\sum\limits^n_{k=1}f(\xi_k)\Delta x_k+\beta\sum\limits^n_{k=1}g(\xi_k)\Delta x_k
$$
Переходя к пределу при $d\to 0$, получим требуемое равенство.
\item $f,g\in\R[a,b]$ Заметим, что $fg=\frac14((f+g)^2-(f-g)^2)$. По свойству 1, достаточно доказать, что если $f\in\R[a,b]$, то и $f^2\in\R[a,b]$.

Предположим сначала, что $sup\{f(x)\}=M>0$ и $m>0$. Пусть $\vs,\ns$ - суммы Дарбу для $f$ по $\{x_k\}$, а $\vs',\ns'$ - для $f^2$.
$$
\vs'-\ns'=\sum(\tilde{M}_k-\tilde{m}_k)\Delta x_k=\sum(M^2_k-m^2_k)\Delta x_k=\sum\underset{\leq2M}{\underbrace{(M_k+m_k)}}(M_k-m_k)\Delta x_k\leq2M\underset{\vs-\ns}{\underbrace{\sum(M_k-m_k)\Delta x_k}}<\eps
$$
при достаточно мелком разбиении, при котором $\vs-\ns<\frac\eps{2M}$.

Пусть теперь $M,m$ не обязательно больше нуля. Рассмотрим $\tilde{f}(x)=f(x)+\lambda$, где $\lambda\in\R$, что $inf\{\tilde{f}(x)\}>0$. По предыдущему рассуждению $\tilde{f}^2$ интегрируема. Но тогда $\tilde{f}^2(x)=f^2(x)+2\lambda f(x)+\lambda^2$ \then $f^2(x)=\tilde{f}^2(x)-2\lambda f(x)-\lambda^2$ интегрируема как линейная комбинация интегрируемых функций.
\item
\begin{enumerate}
\item Пусть $a\leq c\leq d\leq b$.

Рассмотрим любое разбиение $\{x_k\}$ на $[a,b]$, содержащее точки $c$ и $d$.
$$
(\vs-\ns)_{[a,b]}=\sum_{c<x_{k-1}\leq x_k<d}(M_k-m_k(\Delta x_k+\sum_{\underset{=\gamma>0}{\underbrace{[x_{k-1},x_k]\in[a,b]\setminus(c,d)}}}(M_k-m_k)\Delta x_k
$$
При этом
$$
(\vs-\ns)_{[c,d]}=(\vs-\ns)_{[a,b]}-\gamma\leq(\vs-\ns)_{[a,b]}<\eps
$$
при достаточно малой мелкости разбиения $\{x_k\}$.
\item Для любого $c\colon a<c<b$ по предыдущему доказательству $f$ интегрируема на $[a,c]$ и $[c,b]$. Пусть задано размеченное разбиение $\rr,\ c\in\{x_k\}$:
$$
\sigma_{f}[a,b]=\sum^n_{k=1}f(\xi_k)\Delta x_k=\sum_{x_k\leq c}f(\xi_k)\Delta x_k+\sum_{x_{k-1}\geq c}f(\xi_k)\Delta x_k
$$
Переходя к пределу при $d\to0$, получим требуемое равенство.
\end{enumerate}
\item
\begin{enumerate}
\item
$$
f(x)\geq0,\ a\leq x\leq b\ \overset{?}{\Rightarrow}\ \int\limits^b_af(x)dx\geq0
$$
Любая интегральная сумма:
$$
\sigma_f\rr=\sum^n_{k=1}\underset{\geq0}{\underbrace{f(\xi_k)}}\underset{>0}{\underbrace{\Delta x_k}}\geq0
$$
Переходя к пределу при $d\to0$, получим требуемое равенство.
\item $f\in \underset{\mbox{непрерывные}}{\underbrace{C[a,b]}},\ f(x)\geq0,\ a\leq x\leq b,\ \exists c\in[a,b]$, что $\underset{=\gamma}{\underbrace{f(c)}}>0$

По теореме о сохранении знака непрерывных функций, $\exists\ \delta>0$, что $f(x)\geq\frac\gamma2$ в $U_\delta(c)$ (Если $c=a$ или $c=b$, то будем иметь в виду правую или левую окрестность $c$). Тогда
$$
\int_a^b=\underset{\geq0}{\underbrace{\int_a^{c-\delta}f(x)dx}}+\underset{>\frac\gamma22\delta>0}{\underbrace{\int_{c-\delta}^{c+\delta}f(x)dx}}+\underset{\geq0}{\underbrace{\int^b_{c+\delta}f(x)dx}}>0
$$
\end{enumerate}
\item
\begin{enumerate}
\item Пусть $f,g\in\R[a,b],\ f(x)\leq g(x)\ \forall\ x\in[a,b]$. Для доказательством воспользуемся свойством 4.а для функции $\phi(x)=g(x)-f(x)\geq0$.
\item Пусть $f,g\in C[a,b],\ f(x)\leq g(x),\ \exists c\in[a,b]\colon f(c)<g(c)$. Для доказательства воспользуемся свойством 4.б для функции $\phi(x)=g(x)-f(x)\geq0,\ \phi(c)>0$.
\end{enumerate}
\item Пусть $f\in C[a,b],\ f(x)\geq0\ \forall\ x\in[a,b],\ \int\limits^b_af(x)dx=0$. Для доказательства предположим противное, т.е. пусть $\exists\ c\in[a,b]\colon f(c)>0$. Но тогда по свойству 4.б $\int\limits^b_af(x)dx>0$. Противоречие.
\item Пусть $f\in\R[a,b]$. Для доказательства воспользуемся неравенством $||a|-|b||\leq|a-b|$. В частности, $|a|-|b|\leq|a-b|$. Обозначим, как выше, для разбиения $\{x_k\}$ на $[a,b]\ M_k,\ m_k$ -- $sup,\ inf\ f(x)$ на $[x_{k-1},x_k]$.

$\tilde{M}_k,\ \tilde{m}_k$ -- $sup,\ inf\ |f(x)|$ на $[x_{k-1},x_k]$. В силу равенства, $\forall\ x',x''\in[x_{k-1},x_k]\colon||f(x')|-|f(x'')||\leq|f(x')-f(x'')|$. Переходя к $sup$ этих разносткй, получим:
$$
\underset{=\tilde{M}_k-\tilde{m}_k}{\underbrace{\underset{x',x''\in[x_{k-1},x_k]}{sup}\{||f(x')|-|f(x'')||\}}}\leq \underset{=M_k-m_k}{\underbrace{\underset{x',x''\in[x_{k-1},x_k]}{sup}\{|f(x')-f(x'')|\}}}
$$
Рассмотри разности сумм Дарбу: $\vs,\ns$ -- для $f(x)$ на $\{x_k\}$, $\vs',\ns'$ -- для $|f(x)|$ на $\{x_k\}$.
$$
\vs'-\ns'=\sum^n_{k=1}(\tilde{M}_k-\tilde{m}_k)\Delta x_k\leq\sum^n_{k=1}(M_k-m_k)\Delta x_k=\vs-\ns<\eps\ \forall\ \eps>0
$$
По критерию интегрируемости $f\ \exists\ \{x_k\}\colon\forall\ \eps>0\ \exists\ \{x_k\}$ на $[a,b]\colon\vs'-\ns'<\eps$ \then по критерию интегрируемости $|f(x)|$ интегрируема на $[a,b]$.

Рассмотрим интегральные суммы по некоторому размеченному разбиению $(\{y_j\},\{\xi_j\})$:
$$
|\sigma_f(\{y_j\},\{\xi_j\})|=|\sum^N_{j=1}f(\xi_j)\Delta y_j|\leq\sum^N_{j=1}|f(\xi_j)\Delta y_j|=\sigma_{|f|}(\{y_j\},\{\xi_j\})
$$
Переходя к пределу при $d\to0$ в полученном неравенстве, получим требуемое неравенство.
\end{enumerate}
\end{proof}
\mybold{Замечания:}
\begin{enumerate}
\item Обратное утверждение к свойству 7, вообще говоря, неверно. Пример:

Рассмотрим функцию Дирихле: $$D(x)=\left\{\begin{matrix}&-1,&x\in\mathbb{Q}\\&1,&x\in\mathbb{R}\end{matrix}\right.$$
$D(x)$ не интегрируема, так как $\forall\{x_k\}$ на $[a,b]\ \forall\{\xi_k\}$ -- рациональные точки и $\forall\{j_k\}$ -- иррациональные точки \then $\sigma_D\rr=\sum\limits^n_{k=1}(-1)\Delta x_k=-(b-a);\ \sigma_D(\{x_k\},\{j_k\})=\sum\limits^n_{k=1}\Delta x_k=b-a$, причем $\{x_n\}$ может быть любой мелкости \then $\nexists\ \int\limits^b_aD(x)dx$. Однако $|D(x)|=1$ на $[a,b]$ \then $\int\limits^b_a|D(x)|dx=b-a$.
\item Композиция двух интегрируемых на $[a,b]$ функций не обязательно интегируема.
\end{enumerate}
\begin{theor}Первая теорема о среднем для определенного интеграла.

Пусть $f\in\R[a,b],\ g(x)$ не меняет знак на $[a,b]$. Тогда $\exists\ \mu\in[m,M]\colon\int\limits^b_af(x)g(x)dx=\mu\int\limits^b_ag(x)dx$
\end{theor}
\begin{proof}
Пусть $g(x)\geq0$ на $[a,b]$. Так как $m\leq f(x)\leq M$, то $mg(x)\leq f(x)g(x)\leq Mg(x),\ a\leq x\leq b$. Тогда по~свойству 5
$$\underset{=m\int\limits^b_ag(x)dx}{\underbrace{\int\limits^b_amg(x)dx}}\leq\int\limits^b_af(x)g(x)dx\leq\underset{=M\int\limits^b_ag(x)dx}{\underbrace{\int\limits^b_aMg(x)dx}}\eqno{(\#)}$$ Полагая, что $\int\limits^b_ag(x)dx\neq0$, получим $m\leq\underset{=\mu}{\underbrace{\frac{\int\limits^b_af(x)g(x)dx}{\int\limits^b_ag(x)dx}}}\leq M$, то~есть $\int\limits^b_af(x)g(x)dx=\mu\int\limits^b_ag(x)dx$.

Если $g(x)dx\equiv0$, то имеем в $(\#)$ все части равные 0, т. е. $\int\limits^b_af(x)g(x)dx=0$. Поэтому равенство $\int\limits^b_a\underset{=0}{\underbrace{f(x)g(x)}}dx=\mu\int\limits^b_ag(x)dx$ верно при любом $\mu$.

Если $f(x)\leq0$ на $[a,b]$, то рассмотрим $(-g(x))\geq0$. Для нее формула верна: $\exists\ \mu,\int\limits^b_af(x)(-g(x))dx=\mu\int\limits^b_a(-g(x))dx$. Вынесем $(-1)$ и сократим \then ЧТД.
\end{proof}
\mybold{Следствия.}
\begin{enumerate}
\item Пусть $f(x)\in C[a,b]$. Тогда, так как $m\leq\mu\leq M$, то по непрерывности $f(x)$ для $\forall\ \mu,\ m\leq\mu\leq M,\ \exists\ \xi\in[a,b]\colon f(\xi)=M$ \then формула среднего значения имеет вид $\int\limits^b_af(x)g(x)dx=f(\xi)\int\limits^b_ag(x)dx$.
\item Пусть $g(x)\equiv1$ на $[a,b],\ f\in\R[a,b]$. Тогда по теореме 11 $\exists\ \mu,\ m\leq\mu\leq M\colon\int\limits^b_af(x)dx=\mu(b-a)\left(=\mu\int\limits^b_a1dx\right)$. Если в этих условиях $f\in C[a,b]$, то $\exists\ \xi\in[a,b]\colon\int\limits^b_af(x)dx=f(\xi)(b-a)$.
\end{enumerate}
\section{Интеграл с переменным верхним(нижним) пределом}
\begin{opred}
\mybold{Интегралом с переменным верхним(нижним) пределом от $f(x)$} называется функция

$F(x):=\int\limits^x_af(t)dt,\ a\leq x\leq b\ \left(G(x):=\int\limits^b_xf(t)dt,\ a\leq x\leq b\right)$
\end{opred}
\begin{theor}
Если $f\in\R[a,b]$, то $F(x)=\int\limits^x_af(t)dt$ непрерывен на $[a,b]$.
\end{theor}
\begin{proof}
$|\Delta F|=|F(x+\Delta x)-F(x)|=|\int\limits_a^{x+\Delta x}f(t)dt-\int\limits^x_af(t)dt=|\int\limits^{x+\Delta x}_xf(t)dt|\leq\int\limits^{x+\Delta x}
_x|f(t)|dt\leq\left(|f(t)|\leq Q,\ a\leq t\leq b\right)\int\limits^{x+\Delta x}_xQdt=Q\int\limits^{x+\Delta x}_x1dt=Q\Delta x$ \then $F(x)$ непрерывен в $\forall\ x$.

Если $\Delta x<0$, то $|\Delta F|=|-\int\limits^x_{x+\Delta x}f(t)dt|=|\int^{x+\Delta x}_{x}f(t)dt|$.
\end{proof}
\begin{theor}
Пусть $f(x)\in\R[a,b]$ и непрерывна в точке $x_0\in(a,b)$. Тогда $\int\limits^x_af(t)dt=F(x)$ дифференцируема в точке $x_0$, и $F'(x_0)=f(x_0)$.
\end{theor}
\begin{proof}
Из условия непрерывности $f(x)$ в точке $x_0$ следует, что $\forall\ \eps>0\ \exists\ \delta=\delta(\eps)>0\colon x-x_0=\Delta x<\delta$ \then $f(x_0)-\eps<f(x)<f(x_0)+\eps$
$$
\int^{x_0+\delta}_{x_0}(f(x_0)-\eps)dx<\int^{x_0+\delta}_{x_0}f(x)dx<\int^{x_0+\delta}_{x_0}(f(x_0)+\eps)dx
$$
$$
(f(x_0)-\eps)\delta<\int^{x_0+\delta}_{x_0}f(x)dx<(f(x_0)+\eps)\delta
$$
$$
f(x_0)-\eps<\cfrac{\int^{x_0+\delta}_{x_0}f(x)dx}\delta<f(x_0)+\eps
$$
$$
\delta=\Delta x
$$
Таким образом, $f(x_0)-\eps<\cfrac{\int\limits^{x_0+\delta}_{x_0}f(x)dx}\delta<f(x_0)+\eps\ \underset{\Delta x\to0}{\Longrightarrow}\ |\cfrac{F(x_0+\Delta x)-F(x_0)}{\Delta x}-f(x_0)|<\eps$ \then

$\exists\ \underset{=F'(x_0)}{\underbrace{\underset{\Delta x\to 0}{lim}\cfrac{F(x_0+\Delta x)-F(x_0)}{\Delta x}}}=f(x_0)$, т. е. $\exists\ F'(x_0)=f(x_0)$.
\end{proof}
\mybold{Замечание}

Если $f(x)$ непрерывна в точке $a$ справа (в точке $b$ слева), то аналогично можно показать, что $\exists\ F'(a+0)=f(a)\ \left(\exists\ F'(b-0)=f(b)\right)$
\begin{theor}Вторая теорема о среднем для определенного интеграла.

Пусть $f(x)\in\R[a,b]$, а $g(x)$ монотонна на $[a,b]$. Тогда $\exists\ \xi\in(a,b)$, что
$$
\int\limits^b_af(x)g(x)dx=g(a)\int\limits^\xi_af(x)dx+g(b)\int\limits_\xi^bf(x)dx\eqno{(\heartsuit)(\mbox{извините, нет ,,цветочка''})}
$$
\mybold{Замечание.}

Если $g(x)\geq0$ и не возрастает, или $g(x)\leq0$ и не убывает, то $(\heartsuit)$ имеет вид
$$
\int\limits^b_af(x)g(x)dx=g(a)\int\limits^\xi_af(x)dx
$$
\end{theor}
\begin{proof}
Эта теорема без доказательства.
\end{proof}
Следствие теоремы 14.
Если $f(x)\in\mathbb{C}[a,b]$, то $\Phi(\xi)=\int\limits^\xi_af(x)dx$ явдяется ее первообразной. $\Phi'(x)=f(x),\ x\in[a,b]$.
\begin{theor}Формула Ньютона-Лейбница.

Пусть $f\in\mathbb{c}[a,b]$, и $\Phi(x)$ - некоторая ее первообразная. Тогда $\int\limits^b_af(x)dx=\Phi(b)-\Phi(a)=\Phi(x)\raz ab$.
\end{theor}
\begin{proof}
По бредыдущему следствию, у $f(x)$ есть первообразная $F(\xi)=\int\limits_a^\xi f(x)dx$. Тогда $\Phi(x)=f(x)+C$, и $\int\limits^b_af(x)dx=\left(\int\limits^b_a-\int\limits^a_a\right)f(x)dx=F(b)-F(a)=\Phi(b)-C-(\Phi(a)-C)=\Phi(b)-\Phi(a)$
\end{proof}
\begin{remark} Наличие у $f(x)$ первообразной и её интерируемость не следуют друг из друга.

Пример:
\begin{enumerate}
\item $f(x)=\left\{\begin{matrix}x^2sin\frac1{x^2},&x\in(0,1)\\0,&x=0\end{matrix}\right.$

$f'(x)=\left\{\begin{matrix}2xsin\frac1{x^2}-\frac2xcos\frac1{x^2},&x\in(0,1)\\0,&x=0\end{matrix}\right.$


$\int\limits^1_0f'(x)dx$ не существует, т. к. $f'(x)$ не ограничена. Однако существует ее первообразная $f(x)$.
\item $R(x)=\left\{\begin{matrix}0,&x\ -\mbox{иррационален}}\\\frac1n,&x=\frac mn\ -\ \mbox{несократимая дробь}\end{matrix}\right.$ - функция Римана

Можно показать, что у этой функции следующие свойства:
\begin{enumerate}
\item $R(x)$ непрерывна в иррациональный точках
\item имеет разрывы первого рода во всех рациональных точках
\item она интегрируема, и $\int\limits^1_0R(x)dx=0$
\item $R(x)$ не имеет первообразной (если бы она была, то $R(x)=F'(x)$, но тогда не было бы разрывов первого рода)
\end{enumerate}
\end{enumerate}
\end{remark}
\section{Основные приемы вычислния определенного интеграла Римана}
\begin{enumerate}
\item Замена переменной
\begin{theor} Пусть $f(x)\in\mathbb{C}[a,b],\ x=\phi(t)\in\R[a,b]$, и $\phi'(t)\in\mathbb{C}[\alpha,\beta]$, причем $\phi(\alpha)=\underset{\alpha\leq t\leq\beta}{inf}\{\phi(t)\}=a,\newline\phi(\beta)=\underset{\alpha\leq t\leq\beta}{sup}\{\phi(t)\}=b$. Тогда $\int\limits^b_af(x)dx=\int\limits^\beta_\alpha f(\phi(t))\phi'(t)dt$
\end{theor}
\begin{proof}
Пусть $F(x)$ - первообразная для $f(x)$. Тогда $\int\limits^b_af(x)dx=F(x)-F(a)=F(\phi(\beta))-F(\phi(\alpha))$. Покажем, что $F(\phi(t))$ является первообразной для $f(\phi(t))\phi'(t)$. В самом деле, $(F(\phi(t)))'=F'(x)\phi'(t)=f(x)\phi'(t)=f(\phi(t))\phi'(t)$ \then по формуле Ньютона-Лейбница $\int\limits^b_af(\phi(t))\phi'(t)dt=F(\phi(\beta))-F(\phi(\alpha))=F(b)-F(a)=\int\limits^v_af(x)dx$
\end{proof}
\item Интегрирование по частям
\begin{theor} Пусть $u(x),\ v(x),\ u'(x),\ v'(x)\in\mathbb{C}[a,b]$. Тогда верна формула интегрирования по частям:
$$
\int\limits^b_au(x)v'(x)dx=u(x)v(x)\raz{a}{b}-\int\limits^b_av(x)u'(x)dx
$$
\end{theor}
\begin{proof}
Заметим, что $u(x)v(x)$ - первообразная для $f(x)=u'(x)v(x)+u(x)v'(x)$. Тогда по формуле Ньютона-Лейбница имеем:
$$
\int\limits^b_af(x)=\int\limits^b_a(u'(x)v(x)+u(x)v'(x))dx=u(x)v(x)\raz ab\ \Longrightarrow\ \int\limits^b_au(x)v'(x)dx=u(x)v(x)\raz ab-\int\limits^b_av(x)u'(x)dx
$$
\end{proof}
\end{enumerate}
\section{Приложение определенного интеграла}
I. Длина кривой.

\begin{opred}
\mybold{(Непрерывной) плоской кривой} называется ГМТ на плоскости следующего вида: $K=\{(x,y)|x=\phi(t),\ y=\psi(t),\ t\in[\alpha,\beta]\}$, где $\phi(t),\ \psi(t)\in\mathbb{C}[\alpha,\beta]$.
\end{opred}
\begin{opred}
Точка $A(x_0,y_0)\in K$ называется \mybold{кратной для $K$}, если $\exists\ t_1\neq t_2\in[\alpha,\beta]\colon\phi(t_1)=\phi(t_2)=x_0$, $\psi(t_1)=\psi(t_2)=y_0$.
\end{opred}
\begin{opred}
Кривая $K$ называется \mybold{простой}, если у нее нет кратных точек.
\end{opred}
\begin{opred}
Кривая $K$ называется \mybold{простой замкнутой}, если у нее существует единственная кратная точка $A(x_0,y_0)=K(\alpha)=K(\beta)$, и $\forall\ t\neq\alpha,\neq\beta\ K(t)\neq A$.
\end{opred}
\begin{opred}
Кривая $K$ называется \mybold{параметризуемой}, если существует разбиение $[\alpha,\beta]$ на части:

$[\alpha,\beta]=\ind{k=1}n\cup[t_{k-1},t_k]$, где $t_0=\alpha<t_1<\ldots<t_{n-1}<t_n=\beta$, такое, что любой участок $K_k$, соответствующий $[t_{k-1},t_k]$, является простой кривой.
\end{opred}
\begin{opred}
Функция $\phi(t)$ называется \mybold{кусочно-линейной на $[\alpha,\beta]$}, если существует разбиение $\{t_k\},\ t_0=\alpha,\ t_n=\beta$, что $\phi(t)=a_kt+b_k,\ t\in[t_{k-1},t_k];\ a_k,\ b_k\in\R$
\end{opred}
\begin{opred}
Кривая $T$ называется \mybold{ломаной}, если она задана с помощью кусочно-линейных функций на некотором разбиении $\{t_k\}$ отрезка $[\alpha,\beta]$, т.е. $T=\{(x,y)|x=\phi(t),\ y=\psi(t),\ t\in[\alpha,\beta]$, где $\phi(t)$ и $\psi(t)$ -кусочно-линейные функции, соответствующие $\{t_k\}$. При этом точки $T_k$ называются узлами этой ломаной, а участки $T$, соответствующие $[t_{k-1},t_k]$, называются ее звеньями.
\end{opred}
\begin{opred}
Говорят, что \mybold{ломаная $T[\alpha,\beta]$ вписана в кривую $K[\alpha,\beta]$}, если $\forall\ T_k\in K$.
\end{opred}
\begin{remark}
Заметим, что длина ломаной $T$ равна сумме длин ее звеньев.
\end{remark}
\begin{opred}
Длина ломаной $T\colon|T|:=\ind{k=1}n\sum|T_k|=\ind{k=1}n\sum\sqrt{(\phi(t_k)-\phi(t_{k-1}))^2+(\psi(t_k)-\psi(t_{k-1}))^2}$
\end{opred}
\begin{remark}
Заметим, что:
\begin{enumerate}
\item если ломаные $T,\ T'$ вписаны в кривую $K$, и $T'$ получена из $T$ добавлением конечного числа новых узлов, то $|T'|\geq|T|$.
\item $|T|$ не зависит от параметризации. Т.е. если $T=T(\phi(t),\psi(t)),\ t=\gamma(s),\ s\in[s_0,s_n]$, и $\gamma$ - строго монотонная функция, $\gamma(s_k)=t_k\ \forall\ 0\leq k\leq n;\ \tilde{T}=T(f(s),g(s))$, где $f(s)=\phi(\gamma(s)), g(s)=\psi(\gamma(s)).\ |T|=|\tilde{T}|$, т.к. они вычислены через координаты вершин. А вершины у $T$ и $\tilde{T}$ одни и те же.
\end{enumerate}
\end{remark}
\begin{opred}
Кривая $K$ называется \mybold{спрямляемой}, если существует $\ind{T\prec K}{}{sup}|T|=|K|$ - длина $K$.
\end{opred}
\begin{lemma}
Если $K=K_1\ind{[\alpha,\beta]}{}\cup K_2$, где $K_1$ соответствует $[\alpha,\xi)$, а $K_2-[\xi,\beta],\ K$ - простая и спрямляемая, то и $K_1,K_2$ - тоже спрямляемые, и $|K_1|+|K_2|=|K|$.
\end{lemma}
\begin{proof}
Так как $K$ спрямляема, то $\forall\ \eps>0\ \exists\ T\prec K$, что $|T|\leq|K|$, и $|T|>|K|-\eps$. Добавив к $T$ вершину $c=c(\phi(\xi),\psi(\xi))$, получим ломаную $\tilde{T}$, $|\tilde{T}|\geq|T|$ \then $|K|\geq|\tilde{T}>|K|-\eps$. //unimplemented yet и наврядли будет implemented до коллоквиума
\end{proof}
\section{Несобственный интеграл I рода}
\begin{opred}
Пусть $f(x)$ ограничена на $[a,+\infty]$ и интегрируема на любом отрезке $[a,A]$. \mybold{Несобственным интегралом I рода от $f(x)$ на $[a,+\infty]$ называется $\ind{a}{+\infty}{\int}f(x)dx:=\ind{A\to+\infty}{}{lim}\ind{a}{A}{\int}f(x)dx$.} Если этот предел существует и конечен, то говорят, что несобственный интеграл сходится, иначе - расходится.
\end{opred}
\begin{remark}
Аналогично определяется $\ind{-\infty}{a}{\int}f(x)dx$.
$$
\mint{-\infty}{+\infty}=\mlim{B\to-\infty}\mint{B}{a}f(x)dx+\mlim{A\to+\infty}\mint{a}{A}f(x)dx
$$
Все интегалы вида $\mint{a}{+\infty},\mint{-\infty}{a},\mint{-\infty}{+\infty}$ называются несобственными интегралами I рода.
\end{remark}
\begin{remark}
\mybold{Формула Ньютона-Лейбница для несобственных интегралов I рода}

Из определения несобственного интеграла следует, что если $f(x)$ непрерывна на $[a,+\infty],[-\infty,a]$ или $[-\infty,+\infty]$ и $F(x)$ - ее первообразная, то
$$
\mint{a}{+\infty}f(x)dx=\mlim{A\to+\infty}\mint{a}{A}=\mlim{A\to+\infty}(F(A)-F(a))=F(x)\raz{a}{+\infty}
$$
\end{remark}
\begin{opred}\mybold{Интегралом Дирихле I рода} называется следующий интеграл: $\mint{1}{+\infty}\cfrac{dx}{x^\alpha}$
\end{opred}
\begin{remark}
При каких $\alpha$ интеграл Дирихле I рода сходится, а при каких - расходится?
$$
\mint{1}{A}\cfrac{dx}{x^\alpha}=\left\{\begin{matrix}
\cfrac{x^{1-\alpha}}{1-\alpha},&\alpha\neq1\\
ln|x|\raz{1}{A},&\alpha=1
\end{matrix}
\right.\Rightarrow
\mint{1}{+\infty}\cfrac{dx}{x^\alpha}=\left\{\begin{matrix}
\mlim{A\to+\infty}(\cfrac{A^{1-\alpha}}{1-\alpha}-\cfrac1{1-\alpha})=\left\{\begin{matrix}
+\infty,&\alpha<1\\
\cfrac1{\alpha-1},&\alpha>1
{\end{matrix}\right.\\
\mlim{A\to+\infty}(ln|A|-0)=+\infty
\end{matrix}\right.
$$
Вывод: $\mint1{+\infty}\cfrac{dx}{x^\alpha}$ сходится тогда и только тогда, когда $\alpha>1$.
\end{remark}

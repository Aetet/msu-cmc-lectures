\documentclass[final]{report}
\usepackage[utf8]{inputenc}
\usepackage[russian]{babel}

\usepackage{amssymb}
\usepackage{amsmath}
\usepackage{amstext}
\usepackage{amsthm}
\usepackage{amsfonts}
\usepackage{bbold}
\usepackage{hyperref}

%\binoppenalty=10000
%\relpenalty=10000

\hyphenpenalty=10000
\sloppy

\textwidth=180mm
\textheight=227mm
\oddsidemargin=0mm
\headheight=0mm
\hoffset=-10mm
\voffset=-20mm
\parindent=0in

\newcommand{\mbar}{\overline}
\newcommand{\mhat}{\widehat}
\newcommand{\mtilde}{\widetilde}

\newcommand{\argmax}{\operatorname*{arg\,max}}
\newcommand{\w}{\omega}
\newcommand{\nx}{\overline{x}}
\newcommand{\F}{\ensuremath{\forall}}
\newcommand{\E}{\ensuremath{\exists}}
\newcommand{\xor}{\oplus}
\renewcommand{\L}{\mathcal{L}}
\newcommand{\LRA}{\ensuremath{\Leftrightarrow}}

\newtheorem* {lemma} {Лемма}
\newtheorem* {theor} {Теорема}
\newtheorem* {theor*} {Теорема}
\newtheorem* {opred} {Определение}
\theoremstyle {remark}
\newtheorem* {remark} {Замечание}
\theoremstyle {remark}
\newtheorem* {sled} {Следствие}

\newcommand{\s}[1]
{
\newpage
\phantomsection
\section*{\centering #1}
\addcontentsline{toc}{section}{#1}
}

\renewcommand{\ss}[1]
{
\phantomsection
\subsection*{#1}
\addcontentsline{toc}{subsection}{#1}
}

\begin{document}

\tableofcontents
\newpage

\s{[06.09.11] Лекция 1}

\ss {Определение дизъюнктивной нормальной формы (ДНФ). Геометрическая интерпретация ДНФ. Совершенная ДНФ.}

Пусть есть бесконечный алфавит $x_1,x_2,\ldots,x_n,\ldots$. Введем множество $B=E_2=\{0,1\}$.

$B^n=E^n=\{\alpha=(\alpha_1,\ldots,\alpha_n|\alpha_i \in B, i=\overline{1,n}\}$ -- n-мерный булев куб.

$f(x_1,\ldots,x_n): B^n\xrightarrow{} B$ -- булева функция. $x_1,\ldots,x_n$ -- булевы переменные.

Если есть ??? $f(x_1,\ldots,x_n)$, где $x_i$ - булевы переменные, то следует определение формулы над Q.

\begin{opred}
Пусть $\delta$-некоторое множество, а $\delta_1,\ldots,\delta_s$ - некоторое его подмножество. Тогда система подмножеств $\{\delta_1,\ldots,\delta_s\}$ называется покрытием множества $\delta$ тогда и только тогда, когда $\cup_{i=1}^s \delta_i = \delta$

При этом каждое $\delta_i$ называется блоком (компонентой) покрытия $\{\delta_1,\ldots,\delta_s\}.$ 

Покрытие неприводимо тогда и только тогда, когда никакая его компонента не является подмножеством другой его компоненты.
\end{opred}

Пусть $f(\tilde{x}^n)$ - булева функция, тогдв $N_f = \{\alpha=(\alpha_1,\ldots,\alpha_n| f(\alpha)=1\} $ - объект геометрической интерпретации ДНФ.

Пусть $x_i$ - символ переменной, $\delta \in \{0,1\}$, тогда $x_{i}^\delta$ - буква ($x_i$ при $\delta=1, \overline{x_i}$ при $\delta=0$.


\begin{opred}
Элементарная конъюнкция -- это конъюнкция букв различных переменных.

$K=x_{i_1}^{\delta_1}x_{i_2}^{\delta_2}\ldots x_{i_r}^{\delta_r} $ - элементарная конъюнкция. $R(K)=r$ - ранг элементарной конъюнкции.
\end{opred}

\begin{opred}
Элементарной дизъюнкцией называется дизъюнкция букв различных переменных.
\end{opred}

\begin{opred}
ДНФ -- формула, которая представляет собой дизъюнкцию различных элементарных конъюнкций.

КНФ -- формула, которая представляет собой конъюнкцию различных элементарных дизъюнкций.
\end{opred}

\begin{remark}
ДНФ существует тогда и только тогда, когда функция тождественно не равна 0. КНФ существует тогда и только тогда, когда функция тождественно не равна 1.
\end{remark}

\begin{theor} (О разложении булевой функции по переменным)

Пусть $f(x_1,\ldots,x_n)$ - булева функция, $1\leq r \leq n, r\in \mathbb{N}$, тогда $f(x_1,\ldots,x_r,x_{r+1},\ldots,x_n) = \underset {\delta=(\delta_1,\ldots,\delta_n)}{\lor} x_{1}^{\delta_1}x_{2}^{\delta_2}\ldots x_{r}^{\delta_r} f(x_1,\ldots,x_r,x_{r+1},\ldots,x_n)$
\end{theor}
\begin{proof}
Левая часть $\forall \alpha=(\alpha_1,\ldots,\alpha_n) \in B^n, f(x_1,\ldots,x_r,x_{r+1},\ldots,x_n)$ - есть $f(\alpha)$.

Рассмотрим правую часть $\underset {\delta=(\delta_1,\ldots,\delta_n)}{\lor} x_{1}^{\delta_1}x_{2}^{\delta_2}\ldots x_{r}^{\delta_r} f(x_1,\ldots,x_r,x_{r+1},\ldots,x_n)$. \begin{enumerate}
\item $\delta: \delta_i=\alpha_i \ \forall i \in \{1,\ldots,r\}$, тогда соответствующее слагаемое имеет вид (на $\alpha$) $\alpha_{1}^{\delta_1}\ldots\alpha_{r}^{\delta_r}f(\delta_1,\ldots,\delta_r,\alpha_{r+1},\ldots,\alpha_n) = f(\alpha)$
\item $\delta\colon\exists i \in \{1,\ldots,r\} : \delta_i \not= \alpha$, тогда соответствующее слагаемое $ \alpha_{1}^{\delta_1}\ldots\alpha_{i}^{\delta_i}\ldots\alpha_{r}^{\delta_r}f(\delta_1,\ldots,\delta_r,\alpha_{r+1},\ldots,\alpha_n) =0$, т.к. $\alpha_{i}^{\delta_i}=0 $
\end{enumerate}
Итак, правая часть имеет вид $0 \lor 0\ldots0 \lor f(\alpha) \lor 0\ldots.\lor 0=f(\alpha)$
\end{proof}

\begin{sled}
(теорема о СДНФ) Каждая $f(\tilde{x}^n) \not= 0$ имеет место представление $ f(\tilde{x}^n)= \underset {\delta=(\delta_1,\ldots,\delta_n); f(\alpha)=1}{\lor} x_{1}^{\delta_1}\ldots x_{n}^{\delta_n}$
\end{sled}

\begin{remark}
ДНФ рассматривается для функций, зависящих от $x_1,\ldots,x_n$ (если не оговорено противное) 
\end{remark}

\ss {Геометрическая интерпретация ДНФ}

Пусть $\gamma=(\gamma_1,\ldots,\gamma_n)$, где $\forall i \in \{1,\ldots,n\}, \gamma_i \in \{0,1,2\}$

\begin{opred}
Гранью $G_\gamma$ n-мерного булевого куба $B^n$ называется множество $\{\alpha=(\alpha_1,\ldots,\alpha_n)|\alpha_i \in \{0,1\} \ \forall i=1,2,\ldots,n$ и если $\gamma \in \{0,1\}$, то $\alpha_i=\gamma_i\}$

Пусть количество "2" в наборе $\gamma$ есть $n-r$, тогда $r$-ранг грани $G_\gamma$, а $n-r$ - размерность грани. Набор $\gamma$ называется кодом грани $G_\gamma$
\end{opred}

Пример: n=4, $\gamma=(0,2,1,2) G_\gamma = \{(0,0,1,0),(0,0,1,1),(0,1,1,0),(0,1,1,1)\}$

\begin{opred}
$G$-грань булева куба $B^n$ тогда и только тогда, когда существует $\gamma \in \{0,1,2\}^n\colon G=G_\gamma$
\end{opred}
Причем здесь конъюнкция?

Для каждой грани $G_\gamma \in B^n \ \exists!$ элементарная конъюнкция от переменных $x_1,\ldots,x_n$, являющаяся характеристикой функцией этой грани (обозначим эту элементарную конъюнкцию как $K$), то есть $\alpha \in G_\gamma \Leftrightarrow K(\alpha)=1$. Пусть все символы в $\gamma$ не равны 2, суть $\gamma_{i_1},\ldots,\gamma_{i_r}$, значит, искомая элементарная конъюнкция $K$ имеет вид $K=x_{i_1}^{\gamma_{i_1}}\ldots x_{i_r}^{\gamma_{i_r}}$. Ясно, что $N_k=G_\gamma$

Булева функция $ f'$ имплицирует булеву функцию $f''$, если $\forall \alpha: f'(\alpha)=1$ следует, что $f''(\alpha)=1$ (или, по-другому, $f''$ помещает $f'$): $f' \xrightarrow{} f''=1$ или же: $f'*f''=f', f' \lor f'' = f''$.
\begin{opred}
Если элементарная конъюнкция имплицирует f, то говорят, что эта элементарная конъюнкция является импликантой f.
\end{opred}
Пусть $D_f=K_1 \lor \ldots \lor K_s$ - ДНФ, реализующая БК f и $K_1,\ldots,K_s$ - элементарная конъюнкция (f и элементарная конъюнкция от $x_1,\ldots, x_r$). Тогда этой ДНФ соответствует покрытие множества $N_f$ гранями $N_{k_1},\ldots, N_{k_s}$ куба $B^n$. 
\begin{opred}
Элементарная конъюнкция $K$ называется простой импликантой функции $f$, если она не имплицирует никакую другую импликанту $K'$ функции $f$ (то есть $N_k \not \in N_{k'}$)
\end{opred}
\begin{opred}
$G_\gamma$-грань булевой функции f тогда и только тогда, когда $ G_\gamma \subseteq N_f$ ясно, что по умолчанию $G_\gamma$-грань $B^n$.
\end{opred}
\begin{opred}
Пусть $K$ - простая импликанта булевой функции $f$, тогда соответствующая ей грань называется максимальной гранью функции $f$.
\end{opred}
Легко увидеть, что максимальная грань булевой функции $f$ -- это максимальная по включению наборов грань булевой функции $f$.
\begin{remark}
(о совершенной ДНФ): совершенная ДНФ $D_f$ функции f соответствует покрытию $N_f$ нульмерными гранями, то есть точками.
\end{remark}
\begin{opred}
Вес булева набора $\alpha$ - число $||\alpha||$ единиц в нем. $r$-й слой булева куба $B^n$ -- это множество $B_{r}^n=\{\alpha=(\alpha_1,\ldots,\alpha_n)|\alpha \in B^n, ||\alpha||=r\}.$
\end{opred}
\begin{opred}
Два набора называются соседними, если они отличны в одной координате.
\end{opred}

\s{[13.09.11] Лекция 2}

\ss {Сложность ДНФ. Минимальная ДНФ. Кратчайшие ДНФ. Функции Шеннона для ДНФ}

\begin{opred}
Пусть $\phi$-функция, ставящая в соответствие каждой ДНФ некоторым образом число, при этом:
\begin{enumerate}
\item для любой ДНФ $D\colon0 \leq \phi(D)$
\item если ДНФ $D'$ получена из D вычеркиванием букв и слагаемых (элементарных конъюнкций), то $\phi(D') \leq \phi(D)$
\end{enumerate}
В таком случае говорят, что задан неотрицательный функционал $\phi$ сложности (ранга) ДНФ, обладающий свойством монотонности.
\end{opred}
Примеры $\phi(D)$:

$R(D)$ - ранг (сложность) ДНФ $D$, суммарное число букв в $D$.

$\lambda(D)$ - длина ДНФ $D$, число слагаемых в ДНФ $D$.

$L(D)$ - число всех операций, необходимых для построения ДНФ $D$.

\begin{opred}
ДНФ $D'$ называется минимальной относительно функционала $\phi$ булевой функции $f$ ($\phi$ - минимальная ДНФ булевой функции $f$) тогда и только тогда, когда $\phi(D')= \underset {D}{min} \phi(D)$ ($D$ - ДНФ, реализующая $f$).

Ранг минимальной ДНФ называется минимальным, длина - кратчайшей.
\end{opred}

\ss {Функция Шеннона относительно функционала $\phi$ сложности ДНФ}

$\phi(n) =\underset {f(\overline{x}^n) \in P_2}{max} \underset {D}{min} \phi(D)$, $D$ - ДНФ, реализующая $f$.
\begin{remark}
Если $D' - \phi$ -- минимальная ДНФ булевой функции $f$, то говорят, что $\phi(D')$-сложность функции $f$ относительно функционала $\phi$.
\end{remark}
\begin{theor}
Для любого $n\in \mathbb{N}$ имеет место соотношение: $R(n)=n2^{n-1}, \lambda(n)=2^{n-1}$.
\end{theor}
\begin{proof}
(Нижняя оценка) Рассмотрим функцию $f_n(\tilde(x)^n)=x_1\oplus \ldots\oplus x_n$, максимальной грани функции - это точка, следовательно, единственной, следовательно, минимальной ДНФ этой функции является совершенная ДНФ.

Пусть $\alpha', \alpha''$ - соседние наборы в $B^n$, тогда $f(\alpha') \not= f(\alpha'')$, тогда любая максимальная грань есть точка, то есть грань размерности 0, то есть единственная ДНФ $f_n$ есть ее совершенная ДНФ, поскольку $|N_f|=2^{n-1},$ то длина $f_n$ есть $\lambda(f_n)=2^{n-1}$, а ранг $R(f_n)=n2^{n-1}$, следовательно, $2^{n-1}\leq \lambda(n), n2^{n-1}\leq R(n)$.

(Верхняя оценка) Рассмотрим любую функцию $f(\tilde{x}^n) (f \not= 0)$. Разложим $f$ по $x_2,\ldots, x_n\colon f(x_1,\ldots,x_n)=\underset {(\delta_2,\ldots,\delta_n) \in B^{n-1}}{\lor} x_{2}^{\delta_2}\ldots x_{n}^{\delta_n} f(x_1,\delta_1,\ldots,\delta_n) \in \{0,1,x_1,\overline{x_1}\}$, тогда у любой булевой функций $R \leq n2^{n-1}, \lambda \leq 2^{n-1}$, тогда в силу произвольности выбора $f \in P_2(n) \lambda(n) \leq 2^{n-1}, R(n) \leq n2^{n-1}$
\end{proof}

\ss {Тупиковая ДНФ. Сокращённая ДНФ и методы её построения}

\begin{opred}
ДНФ $D=\overset {s}{\underset{i=1}{\lor}}K_i$ называется неприводимой тогда и только тогда, когда покрытие $\{N_{k_1},\ldots,N_{k_s}\}$ является неприводимым.
\end{opred}
\begin{opred}
ДНФ $D$ называется тупиковой тогда и только тогда, когда любая ДНФ $D'$, получающаяся из $D$ вычеркиванием букв или слагаемых, не эквивалентна ДНФ $D$ (то есть $D'$ и $D$ реализуют разные булевы функции).
\end{opred}
\begin{opred}
ДНФ $D$ называется сокращённой ДНФ функции $f$ тогда и только тогда, когда $D$ есть дизъюнкция всех простых импликант булевой функции $f$
\end{opred}
\begin{remark}
Тупиковая ДНФ состоит только из простых импликант.
\end{remark}
\begin{remark}
Тупиковая ДНФ является неприводимой и может быть получена из сокращённой ДНФ выбрасыванием некоторых слагаемых.
\end{remark}
\begin{remark}
Минимальная ДНФ является тупиковой.
\end{remark}
\begin{remark}
Среди всех тупиковых ДНФ есть кратчайшие, но не все кратчайшие ДНФ являются тупиковыми.
\end{remark}
Построение сокращённой ДНФ - это первый этап построения кратчайшей ДНФ.

\ss {Методы построения сокращённой ДНФ}

\begin{enumerate}
\item Геометрический (по определению).
Пусть $N_f=\{\alpha^{(1)},\ldots,\alpha^{(p)}\}$.
\begin{itemize}
\item Шаг 1. Построить покрытие множества $N_f$ гранями $\{\alpha^{(1)}\},\ldots,\{\alpha^{(p)}\}.$
\item Шаг i+1. Дополнить покрытие $N_f$ предыдущего шага всевозможными расширениями на 1 размерность в пределах $N_f$ граней из предыдущего шага; удалить поглощённые грани.
\end{itemize}
За конечное число шагов будет построено покрытие множества $N_f$ всеми максимальными гранями $f$. 
\item Алгоритм Квайна (построение сокращённой ДНФ по КНФ)

Приведение подобных (после раскрытия скобок с ДНФ) предполагает:
\begin{enumerate}
\item приведение слагаемых к виду элементарной конъюнкции или удаление слагаемых
\item применение правил: $K' \lor K'' = K'' \lor K', (K' \lor K'')\lor K'''=K'\lor(K'' \lor K'''), K' \lor K'K''=K'$
\end{enumerate}
Эти преобразования выполняются пока возможно, то есть результат есть ДНФ, в которой $K' \lor K'=K', K' \lor K'K''=K'$ нельзя применить слева направо (при $\forall$ применение тождеств ассоциативности и коммутативности) 
\begin{theor} 
Пусть $D',\ D''$ - сокращённые ДНФ $f',\ f''$ соответственно, тогда ДНФ $D$, получаемая в результате раскрытия скобок и приведения подобных в формуле $D'D''$ является сокращённой ДНФ функции $f=f'f''$.
\end{theor}
\begin{proof}
Достаточно доказать, что любая простая импликанта $K$ функции $f$ является слагаемым $D$.
Так как $K$ -- импликанта, то $K$ имплицирует $f'$ и $K$ имплицирует $f''$, следовательно, так как $D',\ D''$ -- сокращённые ДНФ $f',\ f''$, то существует элементарная конъюнкция $K'$ (слагаемое в $D'$) и элементарная конъюнкция $K''$ (слагаемое в $D''$) - $K$ имплицирует их, следовательно, $K$ имплицирует $K'K''$, но при раскрытии скобок в $D'D''$ и приведении подобных найдется элементарная конъюнкция $\tilde{K}$ (слагаемое в $D$), которую имплицирует $K'K''$, следовательно, $K$ имплицирует элементарную конъюнкцию $\tilde{K}$, но $K$ -- простая импликанта $f$, значит, $K=\tilde{K}$, то есть $K$ встречается в D.
\end{proof} 
\begin{sled}
(описание алгоритма Квайна): сокращённая ДНФ $D$ булевой функции $f$ может быть получена из произвольной КНФ функции $f$ (при $f \not\equiv 1$) путем последовательных раскрытий скобок и приведения подобных.
\end{sled}
\item Метод Блэйка (Нэльсона) (построение сокращённой ДНФ по любой ДНФ).

Правило обобщённого склеивания: $(xK' \lor \overline{x}K''=xK' \lor \overline{x}K'' \lor K'K'')$ применяем пока возможно. Затем приводим подобные.
\begin{opred}
ДНФ $D'$, полученная из ДНФ $D$ применением к каким-то парам ее слагаемых правил обобщённого склеивания, называется расширением ДНФ $D$.
\end{opred}
\begin{opred}
Расширение $D'$ ДНФ $D$ называется строгим расширением ДНФ $D$ тогда и только тогда, когда в $D'$ есть слагаемое (элементарная конъюнкция), не имплицирующее никакое слагаемое в $D$.
\end{opred}
\begin{remark}
Сокращённая ДНФ не имеет строгих расширений.
\end{remark}
\begin{theor}
Пусть ДНФ $D$ является неприводимой и не имеет строгих расширений, тогда $D$ есть сокращённая ДНФ.
\end{theor}
\begin{proof}
Достаточно доказать, что любая простая импликанта встречается в $D$. 

От противного: Пусть $D$ - неприводимая ДНФ булевой функции $f$, не имеет строгих расширений, $K$ -простая импликанта $f$ и $K$ - не слагаемое $D$.
Построим множество $\chi$ всех элементарная конъюнкция, которые являются импликантами $f$, но не имплицируют никакое слагаемое из $D$. $\chi \not\equiv \oslash$, ибо $K \in \chi$.
В таком случае пусть $K$ - элементарная конъюнкция максимального ранга из $\chi, f=f(x_1,\ldots,x_n)$, следовательно, $R(\widehat{K})<n$ (так как если бы $ R(\widehat{K})=n$, то $\widehat{K}$ имплицировала бы некоторые элементарные конъюнкции из D), тогда пусть $x_i$ не встречается в $widehat{K}$, тогда так как $\widehat{K}$ - элементарная конъюнкция максимального ранга из $\chi$, то существует $K', K''\in\chi: x_i\widehat{K}$ имплицирует слагаемое $x_iK'$ из D $\overline{x_i}\widehat{K}$ имплицирует слагаемое $\overline{x_i}K''$ из D.
Следовательно, $\widehat{K}$ имплицирует $K'K''$, которое получается правилом обобщённого склеивания $x_iK'$ и $\overline{x_i}K''$, значит, $\widehat{K} \notin \chi$, получаем противоречие, значит, $K \in D$ 
\end{proof}
\begin{sled}
(метод Блэйка): сокращённую ДНФ можно построить, применяя, пока возможно, правило обобщённого склеивания всех возможных пар слагаемых ДНФ по всем переменным последующим приведением подобных.
\end{sled}

\s{[20.09.11] Лекция 3}

\item Алгоритм, основанный на картах Карно.

Если рассмотреть наборы из 2 аргументов, то их можно упорядочить по коду Грея (расстояние между соседними равно 1, например, 00-01-11-10). Множество $\sim$ тор, любому треугольнику на торе соответствует максимальная грань.
\end{enumerate}

\ss {Сложность одновременного нахождения min и max массива}

За наименьшее число сравнений. Сколько в худшем случае? $(n-1)+(n-2)=2n-3$ 

Нельзя ли проще? Вспомним теорему Мура: $n-1 \leq \alpha(n)$ 

Алгоритм:

разбить элементы на пары, max высшей лиги, min низшей.

$n=2r\colon r+(r-1)+(r-1)= \frac{3n}{2}-2$

$n=2r+1\colon r+(r-1)+(r-1)+2= 3r=\frac{3(n-1)}{2}$

Получаем $\alpha(n) \leq \rceil \frac{3n}{2}-2 \lceil$ (целое сверху)

Минимизация числа сравнений по множеству всех алгоритмов решений.

Алгоритм:

$E(x)=\left\{\begin{matrix}
2, \text{если x может быть max и min} \\
1, \text{если x может быть max илм min} \\
0, \text{если x не может быть ни max, ни min} \\
\end{matrix}\right. $

Энергия массива в начале = 2n, в конце = 2. $E(\overline{x})=\sum_{i=1}^n E(x)$

Энергия массива может уменьшиться на 2. Тогда за $\left]\frac{3n}{2}-2 \right[$ операции можно получить в лучшем случае.

\s{[27.09.11] Лекция 4}

\ss {Выведение сокращённой ДНФ из совершенной ДНФ}

Все простые импликанты входят в нее тогда и только тогда, когда сокращённая ДНФ $K_1\lor \ldots \lor K_n$ простая импликанта $\sim$ нельзя удалить букву из $K_i \leq f$.

Пусть существует сокращённая или иная ДНФ реализующая нашу функцию: $x_1\overline{x_2}\lor x_2\overline{x_3}\lor x_3\overline{x_1}\lor x_1\overline{x_3}\lor x_3\overline{x_2}\lor x_2\overline{x_1}.$

Удаляя по ребру можно получить 5 вариантов. Для покрытия в таких точках необходимо 3 ребра, значит, 2+3 варианта.
\begin{opred}
Тупиковая ДНФ - это ДНФ, к которой неприменимо ни одно из преобразований: удаление буквы или удаление конъюнкции.
\end{opred}
У нашей функции их 5. Вопрос: сколько их может быть?

Ведь из совершенной ДНФ выводится сокращённая ДНФ.

$f(\tilde{x}^n)=(x_1\lor x_2\lor x_3)(\overline{x_1}\lor \overline{x_2}\lor \overline{x_3}) \oplus x_4 \oplus\ldots\oplus x_n$. 

Что можно сказать о допустимых конъюнкциях?

В каждую конъюнкцию этой функции должен входить множитель $x_4^{\sigma_4}\ldots x_n^{\sigma_n}$ нет конъюнкции без этих переменных. $\sigma_4\ldots\sigma_n$ делятся на 2 части: $\sum \sigma_i = 0 ( \text{тупиковая ДНФ? их } 5^{2^{n-4}}), \sum \sigma_i = 1$

\ss{ДНФ Квайна (основана на построении таблицы Квайна)}

$K_1(\overline{\alpha_1})=1$

$K_1(\overline{\alpha_i}=0)$, следует решение задачи о покрытии, имея о том, кого выкинуть из сокращённой ДНФ. Сначала решаем, кого оставить; нельзя выкинуть конъюнкцию, которая единственная покрывает некоторый набор (такая конъюнкция называется ядром), тогда и только тогда это ДНФ Квайна.

\begin{theor}(Журавлева)
\begin{enumerate}
\item Простая импликанта $K$ входит в ДНФ $\sum T$ тогда и только тогда, когда $K$ не является регулярной.
\item Конъюнкция $K$ называется регулярной тогда и только тогда, когда все ее точки регулярны относительно этой конъюнкции
\item $\tilde{\alpha}$ регулярна относительно $K$, если существует $\tilde{\beta}: f(\tilde{\beta})=1, K(\tilde{\beta})=0, \Pi(\tilde{\beta})\leq\Pi(\tilde{\alpha}) (\Pi(\tilde{\beta}) - \text{множество} K_i$, которые обращаются в 1 на $\beta$, тогда все $K_i: K_i(\beta)=1$, тогда $K_i(\alpha)=1$.
\end{enumerate}
\end{theor}
\begin{proof}
=>(от противного) Почему $K$ не принадлежит тупиковой ДНФ.

Пусть $K$ регулярна и обращается в 1 на наборах $\tilde{\alpha_1},\ldots,\tilde{\alpha_m}$, тогда $\ \exists \tilde{\beta_1},\ldots,\tilde{\beta_n}: f(\tilde{\beta_i})=0, K_m '(\tilde{\beta_m})=1$.

Рассмотрим $K_1',\ldots,K_m': K_1'(\tilde{\beta_1})=1,\ldots, K_m'(\tilde{\beta_m})=1$, значит, покрываются все такие точки конъюнкции $K$, следовательно, её можно выкинуть.

<= Если $K$ не является регулярной, то входит ли она в тупиковую ДНФ? Проотрицаем определение регулярности.

$K$ является нерегулярной, если существует точка, которая не является регулярной.

$\tilde{\alpha}$ нерегулярна относительно $K$, если $\forall \tilde{\beta}: f(\tilde{\beta})=1, K(\tilde{\beta})=0$, следовательно, $\Pi(\tilde{\alpha})\leq\Pi(\tilde{\beta})$

Пусть $\tilde{\alpha}$ - нерегулярная точка.
Рассмотрим ДНФ $K_1'\lor\ldots\lor K, \text{где } K_i'$ - все простые импликанты: $K_i'(\tilde{\alpha})=0$.
Эти ДНФ реализует функция.
Для всех $\tilde{\beta}$ ее пучок не принадлежит пучку $\tilde{\alpha}$. Делаем тупиковую, но $K$ выбросить нельзя, ибо она единственная, которая обращается в 1 на $\tilde{\alpha}$.
\end{proof}
Рассмотрим несколько функций. Ясно, что на булевом кубе можно уложить цикл и цепочку только четной длины.

Соседние наборы на булевом кубе имеют разные четности, тогда, пройдя по циклу, мы сменили бы четность. Кроме того, булев куб двухцветен, а нечетный цикл трёхцветен. Минимальная ДНФ для цикла цепочки четной или нечетной длины $K_1,\ldots, K_n$.

В цепочке в любом случае мы должны брать хвосты (а не ядра). Если числа наборов четные (соответственно число звеньев нечетно), то $K_1\lor K_2\lor K_3$. Если нечетная длина, то минимальная ДНФ должна содержать (2n+1 наборов) n+1 конъюнкцию, то переход с четной на нечетную в каком-либо месте.

Пусть наш алгоритм не может перелезть через некоторый параметр. II теорема Журавлева (сумма минимальных не лежит в классе локальных алгоритмов).


\ldots \ldots \ldots \ldots \ldots \ldots \ldots

Есть некоторая пара ("управляющая система"): $<\Sigma, f>, \Sigma$ - функция, или функционирование. $\alpha(\Sigma)$ - функционал сложности.

Мы хотим найти $L(f)=\underset{f(x_1,\ldots,x_n)}{max}\underset{\Sigma, f}{min} \alpha{f}$.

Аргумент функции Шеннона - число переменных.

\s{[04.10.11] Лекция 5}


Оценим $L(n)$ сверху: произв. схема синтеза; снизу: оценка основана на том, что едва схемы f функцию не реализуют.

$L(n)\leq n+(n-1)2^n + 2^n - 1$

Для улучшения надо ввести понятие дешифратора $D_n: L(n)\leq 2^n - 1 + L(D_n)$. Сложность дешифратора асимптотически равна $2^n$.

Пусть есть дешифратор $D_{n-1}$, как получить из него $D_n$? Берем $x_n $, $ x_n x_n\ \forall \text{на} K_i$, тогда $L(D)\leq L(D_{n-1})+1+2^n; L(D_1)=1, L(D_2)=6\ldots$

$L(D_n)=L(D_{n-r})+L(D_r)+2^n$

\ss {Разложение функции по $k$ переменным}

$f(x^n)=\underset {\sigma=(\sigma_1,\ldots,\sigma_r)}{\lor} x_1^{\sigma_1}\ldots x_r^{\sigma_r}f(\sigma_1,..,\sigma_r,x_{r+1},\ldots,x_n)$

$L(n)\leq L(D_r)+L(U_{n-r})+2^r+2^{r-1}$

$L(U_n)=2^{2^n}$, тогда $L(n)\leq L(D_r)+L(U_{n-r})+3\cdot2^{r-1}\leq 3\cdot 2^{r-1}+O(2^{r/2})+2^{2^{n-r}}$

Как оптимально выбрать r? $3\cdot 2^r ln(r) + 2^{n-r}\cdot2^{2^{n-r}}ln(n-r)=0$

$r\approx2^{n-r}$

Рассмотрим $r=\left]n-log_2 n\right[$

$n-log_2 n = 2^{log_2 n}=n$

$3\cdot 2^{n-log_2 n} + 2^n = 2^n(3\cdot2^{-log_2 n}+1)$

Если $r=n-log_2 log_2 n $

$3\cdot 2^{n-log_2 log_2 n}+2^{2^{log_2 log_2 n}}=3\cdot 2^{n-log_2 log_2 n}+2^{log_2 n}$

Если $r=n-log_2(n-2log_2 n)$

$3\cdot2^{n-log_2(n-2log_2 n)} + 2^{n-2log_2 n} = 2^n(3\cdot2^{-log_2(n-2log_2 n)}+4n^{-2})=2^n(3(n-2log_2 n)^{-1}-n)^{-2}$

$2^r=2^{]n-log_2(n-2log_2 n)[} \leq \frac{2^{n+1}}{n-2log_2 n}$

$L(n)\leq 3\cdot 2^r +O(R^{r/2})+ 2^{2^{n-r}}$

При $r=]n-log_2(n-2log_2 n)[=O(2^n/n)+O(2^{r/2})+O(2^n/n)$

Получим оценку $\Theta(2^n/n)$


\ss {Нижняя оценка функции Шеннона}

Сколько всего можно получить схем сложности L (от n переменных)?
Связной является схема, реализующая функцию.
Утверждения про связные графы:\begin{enumerate}
\item количество нечетных вершин четно
\item у связного графа можно выделить остовное дерево
\end{enumerate}

Выделим остовное дерево функции в базисе $\{\&,\lor,\neg\}$

Если сложность L, то вершин максимально n+L. Сколько различных деревьев с n+L вершинами: $L+n-1\ldots4^{L+n-1}$

Будут некоторые вершины, которые внутренние вершины, то есть L вершин, в которых неизвестны элементы, тогда $4^{L+n-1}n^{L+n}3^L(L+n)^L L$

Требование: $\varepsilon 2^{2^n}\leq \ldots$

$2^n - log_2 \varepsilon \leq 2(L+n-1)+(L+n)log_2 n +Llog_2 3+Llog_2(L+n)+log_2 L$

Пусть $L \leq (1-\delta)\cdot 2^n/n$

при $n\to\infty: 1,2,3,5 = o(2^n-log_2 \varepsilon)$

$(1-\delta) \frac{2^n}{n}log_2(n+(1-\delta)2^n/2)$

$(1-\delta) \frac{2^n}{n}n(1+o(1))$

Эффект Шеннона: для почти всех функции $n \to \infty$ сложность почти $\frac{2^n}{n}$

\s{[11.10.11] Лекция 6}

\ss {Метод Лупанова для СФЭ}

Идея представления булевой функции в виде прямоугольной таблицы.
Этот прямоугольник режется на равные полосы шириной $S$.
Сколько их?
$\left]2^r/s\right[=p S'$ (последний) может быть меньше, поэтому в нашей задаче 3 параметра $r,\ s,\ n$.

Будет предложен способ ее решения, после чего будет необходимо оценить, насколько сложна схема, полученная с помощью этого решения. Мы хотим реализовать короткий стб, то есть мал. функцию, завис.

\ldots

Пусть у нас есть дешифратор, во что это обойдется? S-1 дизъюнкция.

Длинный стб - собирается из коротких стб, сложность P-S дизъюнкций.

$f(x_1,\ldots,x_n)=\underset{\gamma_1,\ldots,\gamma_r}{\lor}x_1^{\gamma_1}\ldots x_r^{\gamma_r} f(\gamma_1,\ldots,\gamma_r, x_{r+1},\ldots,x_n)$ - Шенноновское разложение, будем брать:

$f(x_1,\ldots,x_n)=\underset{\gamma_{r+1},\ldots,\gamma_n}{\lor}x_{r+1}^{\gamma_{r+1}}\ldots x_n^{\gamma_n} f(x_1,\ldots,x_r,\gamma_{r+1},\ldots,\gamma_n)$

Теперь мы сделаем еще один шаг.
Метод Лупанова - м-д 2 порядка Шеннона, но дает оконч. ответ.

Будем собирать длинный стб из коротких. Нам необходимо n, затем нам необходимо получить дешифраторы $D_k, D_{n-k}\colon n+2(2^r+2^{n-r})+\ldots$ нужно построить все функции коротких стд: $2^s p(s-1)$, потом длинные: $2^{n-r}(p-1)+2\cdot2^{n-r}$ на сборку.

Естественно предположим, что $s, r, 2^r/s \to \infty$. 

$p\sim 2^p/s; 2^{n-r}(p-1)\leq2^{n-r}2^r/s \sim 2^r/s; p2^s(s-1)=]2^r/s[2^s(s-1)\sim2^{r+s}$

Очевидно, $s=n(1-o(1))$. С другой стороны, казалось бы, r можно подбирать как угодно, но не так: $p \to \infty; 2^r/s \to \infty, \text{если } \ln s = o(r)$

2 или 3 логарифма: если $r=[3log n]; s=[n-5log_n]$, то $2^{n-r}(p-1)=2^n/s \sim 2^n/n; 0,5n^3\leq2^r; s\leq n$, тогда $p2^s(s-1)\sim2^n/n^2$, тогда главным слагаемым является первое.

Итак, мы доказали, что для любой функции по методу Лупанова можно построить схему некоторой сложности, которая при $n \to \infty$ стремится к $2^n/n$.

$lim \frac{\text{лев}}{\text{прав}}<1 $; левая асимптотически не превосходит правую.

Если есть нижняя оценка Шеннона и решение Лупанова, то задача, возник. реальна: надо синтезировать некую функцию: у нас есть оценка для п.в. и оценка наихудш.

\ss {Контактные схемы (КС)}

\s{[18.10.11] Лекция 7}

\ss {Эквивалентные преобразования}

Пусть есть $<\Sigma',f>, <\Sigma'',f>$. 

Задача синтеза: построить систему, обладающую заданными свойствами.

Задача анализа: понять, что описано.

Упрощение: вводим $L(\Sigma)$ и минимизируем.

ЭВП:$<\Sigma',f>\leftrightarrow<\Sigma'',f>$

\ss {Алгоритм задачи неразрешимости самоприм.}

Базис $\{ \&,\lor,\neg,0,1\}$, мы хотим выписать систему тождеств.
Если мы хотим привести куда-либо $<\Sigma',f>,<\Sigma'',f>$, неплохо было бы привести и онозначно опред. каноническому виду $<\Sigma''',f>$

$(x_1x_2)x_3, x_2(x_1x_3)$ - разные для ЭВП.

Какие нужны преобразования, чтобы разные представления ДНФ приводились к единому виду?
Их 4: коммутативность, ассоциативность, конъюнкция, дизъюнкция.
В формуле может появиться 0, то есть нам нужно еще $x\overline{x}=0$

Итак, нам нужно выражение $K_1\lor\ldots\lor K_s$, а это совершенная ДНФ.

$0\lor x =x; 1\lor x =1; x\lor \overline{x}=1; x\overline{x}=0; xy\lor x\overline{y}=x; x(y\lor z)=xy\lor xz; x\cdot x=x; x\lor x=x; 0\cdot x=0; \overline{\overline{x}}=x$

\begin{enumerate}
\item Пусть $K$ - замкнутый класс, $A$ - базис в нем. В качестве $K$ можно взять $K=[A]$
\item Пусть есть нек. В: $[B]=K\ $, тогда $[B]=[A]$
\item Для А существует конечная полная система тождеств (КПСТ), тогда для В также существует КПСТ.
\end{enumerate}

$F_1 \sim F_1',\ldots, F_m \sim F_m'$ - ф-лы, в базисе А они являются эквивалентными формами.

Можем выписать А и В: $A=\{g_1(x),\ldots,g_r(x)\}; B=\{h_1(x),\ldots,h_l(x)\}$. Все функции g выражаются через h:

$g_1(\tilde{x})=G_1(h_1(\tilde{x}),\ldots,h_l(\tilde{x})),\ldots, g_r(\tilde{x})=G_r(h_1(\tilde{x}),\ldots,h_l(\tilde{x}))$ и

$h_1(\tilde{x})=H_1(g_1(\tilde{x}),\ldots,g_r(\tilde{x})),\ldots, h_l(\tilde{x})=H_l(g_1(\tilde{x}),\ldots,g_r(\tilde{x}))$ 

Наличие КПСТ не зависит от выбранного базиса. (Мы здесь не пояснили, о чём говорим, ибо это неважно!)

%$f_n(x_1,\ldots,x_n)=\left\{\begin{matrix}1,& x_1=\ldots=x_n=2\\0,& \mbox{иначе}\end{matrix}\right.$
%
%$f_n(x_1, f_n())=0$
%
%$[0,f_1,\dots,f_n,\ldots]=\{0,f_1,\ldots,f_n,\ldots\}$
%
%Что здесь будет базисом?
%
%$f_{n-1}(x_1,\ldots,x_n-1)=f_n(x_1,\ldots,x_{n-1},x_{n-1})$
%
%Б\'{о}льшая функция выбивает меньшую из системы. Пусть $l>n$.
%
%$f_n(\ldots)$\{где-то 2 ист.\}$\equiv 0$

%Б\'{о}льшую из меньшей получить нельзя.

\s{[01.11.11] Лекция 8}

Ещё о функции Линдана.

\begin{enumerate}
	\item $ x \cdot x = 0 $
	\item $ x \cdot (y \cdot z) = 0 $
	\item $ x_1 \cdot \cdots \cdot x_n \cdot x_1 = 0 $
	\item $ x_1 \cdot x_2 \cdot \cdots \cdot x_n \cdot x_2 =
		 x_1 \cdot \cdots \cdot x_n $
\end{enumerate}

Все переменные функции существенны.

\ss {Тесты для таблиц}

"Чёрный ящик", например.

Первое формальное определение в работе Яблонского: есть матрица (таблица), мы хотим определить <по значениям> на нём наборе строк.

Задача проверяющего тестирования и диагностического тестирования.

(1) эталонный отб. или нет

(2) где, что именно есть промежуточные

Боремся за длину теста (кол-во строк) стираемая седал минимальной.
$ L(n), T(n) $

Вопрос: существует матрица из $ n $ сб. кроме эталонного. Что можно сказать о длине проверяющего теста?

Не менее 1, не более n.

А диагностический тест?

\begin{theor}
	Длина минимального диагностического теста удовлетворяет
	$ \log_2 n \leqslant J(n) \leqslant n -1 $
\end{theor}

Мы хотим угадать какое-либо его вариантов ответа не менее $ n $.

$ n - 1 $: это доказано в теореме Мура. Bзначально есть 1 класс эквивалентности $ R(0) = 1, R(T) = n $, бессмысленно брать строку, увеличивает число классов эквивалентности.

Если перед ними случ. матрица, то какое ??? ?

\begin{theor}
	Для почти всех т-ц. первые $ 2 \log_2 n + \phi(n) $ строк являются
	диагностическими тестом, если $ \phi(n) \to \infty $
	(как угодно медленно)
\end{theor}

\begin{proof}
	$ m \times n $ матрица, мы хотим разные столбцы. Сколько таких матриц?
	$$
		2^m (2^m - 1) \ldots (2^m - n + 1)
	$$
	Всего матриц $ 2^{mn} $
	$$
		p = \frac{2^m (2^m - 1) \ldots (2^m - n + 1)}{2^{mn}} \geqslant
		 \frac{(2^m - n)^m}{2^{mn}} \geqslant
			(1 - \frac{n}{2^m})^n
	$$

\end{proof}

\s{[08.11.11] Лекция 9}

Есть схема Карно. Требуется построить проверяющий тест относительно разложения.

"Взрослая задача": надо найти ответ самим и доказать его потом.

$ n \geqslant 3 $ не произошло ли размыкание?

$ f(10 \ldots 00) = 1 $
Если $ f'(10 \ldots 00) = 0 $, то разомкнулись верхние контанты.

Можно взять $ f(00 \ldots 01) $ и наборы $ (11 \ldots 1 (n \mod 2)) $.

Почему нельзя обойтись 3 наборами.

А что с замыканием?

Возьмём набор, на котором $ f(\ldots) = 0 $.

$ f(0 \ldots 0) = 0 $ — проводимость проверяется наверху или снизу

$ f(1 \ldots 1) = 0 $, $ n $ чётно

если $ n $ нечётно: $f(0,1 \ldots 1)$ и $f(1 \ldots 1,0)$ то 2 или 3 набора.

\ss {Полный диагностический тест для КС}

\F\ безусловный ПДТ для КС, реализующий
$ x_1 \xor x_2 \xor \ldots \xor x_n $, содержит все наборы.

Как устроена схема, реализующая $ \bigoplus x_i $?
\[
	x_i^{\sigma_1} \ldots x_n^{\sigma_n}, \quad
	\sigma_1 \oplus \ldots \oplus \sigma_n = 1
\]

Если разорвать всё кроме этой цепи, то схема б. реализ-ть
$ x_1^{\sigma_1} \land \ldots \land x_n^{\sigma_n} $.

Если $ \sigma_1 \oplus \ldots \oplus \sigma_n = 0 $, то есть срез ???.
Мимо них нельзя пройти. Если замкнуть всё кроме них, то б.
$ x_1^{} \lor \ldots \lor x_n^{\sigma_n} $

Из этой схемы можно получить \textbb{0}, \textbb{1} (замкнуть или разомкнуть
всё).

Если в тест не вкл. набор ??? к.-п. набор $ \bigwedge x_i^{\sigma_i} $,
то не отличим от \textbb{0}, если не вкл. $ \bigvee x_i^{\overline{\sigma_i}} $, то
не отличим от \textbb{1}.

! Если схема Карно из 4-x переменных, нужно построить полный диагностический
тест (условный). Нужны все наборы или нет? Можно ли за 15 наборов?

Каким образом решается задача о тестировании матрицы?

$$
\begin{pmatrix}
	f_0 & f_1 & \cdots & f_n \\
	\vdots & \vdots & \ddots & \vdots
\end{pmatrix}
\Rightarrow
\begin{pmatrix}
	f_0 \oplus f_1 & f_0 \oplus f_2 & \cdots & f_0 \oplus f_n \\
	\vdots & \vdots & \ddots & \vdots
\end{pmatrix}
$$

Надо подобрать такие строки, чтобы 1 столбец отличался от всех других.

Найти строки: в любом столбце есть единицы.

< Задача о полноте и базисе в $ P_2 $ >

Задача о покрытии матриц.

$$
\begin{pmatrix}
	f_1 & \cdots & f_n \\
	\vdots & \ddots & \vdots
\end{pmatrix}
\Rightarrow
\begin{pmatrix}
	f_1 \oplus f_2 & \cdots & f_1 \oplus f_n & \cdots & f_{n-1} \oplus f_n \\
	\vdots & & \ddots & & \vdots
\end{pmatrix}
$$

Строим все тупиковые покрытия матрицы. Для этого: есть строки $ i, j, k $,
где 1 у $ f_1 $
\[
	( i \lor j \lor k ) ( k \lor l \lor m ) \ldots ( \ldots ) = 1
\]
в каждом столбце матрицы №4.

После этого формальное раскрытие скобок и упрощение (\'a la алгоритм Нельсона).

\[
\bordermatrix{
	 & & & & & & \cr
	1 & 1 & 1 & 0 & 0 & 0 & 0 \cr
	2 & 0 & 1 & 1 & 0 & 0 & 0 \cr
	3 & 0 & 0 & 1 & 1 & 0 & 0 \cr
	4 & 0 & 0 & 0 & 1 & 1 & 0 \cr
	5 & 0 & 0 & 0 & 0 & 1 & 1 \cr
	6 & 1 & 0 & 0 & 0 & 0 & 1 \cr
}
\]
\begin{multline*}
	(1 \lor 6)(1 \lor 2)(2 \lor 3)(3 \lor 4)(4 \lor 5)(5 \lor 6) = \\
	(1 \lor 26)(3 \lor 24)(5 \lor 46) = \\
	135 \lor 246 \lor 1245 \lor 1346 \lor 2356
\end{multline*}
тупиковые ДНФ для функции, которая = 0 на 3 и более.


\ss {Градиентный алгоритм}

Сформулируем задачу.
Что такое \emph{градиентный алгоритм} (жадный):
минимизируем функции по антиградиенту.

Проблемы: с $ f'' $, которая может быстро меняться.

Есть параметр, который связан с оптимизацией и который мы меняем.


\ss {Градиентный алгоритм для задачи о покрытии}

Состоит в том, что берём строку, которая покрывает наибольшее количество
столбцов и убираем всё вместе со столбцами и т. д.

При определённых условиях этот алгоритм работает качественно.
\begin{theor}
Если в булевой матрице $ m \times n $ доля единиц в любом столбце не меньше $p$, то градиентное покрытие не более $ 1 - \log_{1 - p}(n) $ строк.
\end{theor}
\begin{proof}
Выбрав первую строку, имеем в ней же менее $ p $ доли столбцов.

$ '1' \geqslant pmn $ разбрасываем по $ m $ стр.

Теперь осталось $ \leqslant (1 - p)n $ столбцов, следовательно, доля единиц повысилась.
$$
	(1 - p)^{t}n, \quad t \text{ — ?} \qquad (1 - p)^{t} \leqslant \frac{1}{n},
	\text{ то есть } t \geqslant - \log_{1-p}n
$$
\end{proof}

\textbf{Пример}: дискретная задача, где алгоритм работает хорошо.
Поиск наикратчайшего остовного дерева.
Задан связный граф, ребра с весами. Найти остовное дерево с наименьшим
суммарным весом рёбер. (Будем считать, что граф полный и что веса $> 0 $.)

\textbf{Алгоритм}: беру ребро наименьшего веса. Беру следующие лёгкие, если они не образуют цикл...
\begin{proof}
	$ l_1, \ldots, l_{n-1}\ \Rightarrow$ дерево (алгоритма)
	$ = D(\text{алг}) $ \\
	$ l_1, \ldots, l_p, l_{p+1}, \ldots l_{n-1} $ и \\
	$ l_1, \ldots, l_p, l_{p+1}, \ldots l_{n-1} $
	— $ \widetilde{D}(\text{опт}), \exists $, если $ D(\text{алг}) $
	неокб ??? \\
	р/м $ l_1, \ldots, l_p, l_{p+1}, l'_{p+1}, \ldots, l'_{n-1} $
	— граф с 1 циклом.
\end{proof}

Если есть $l'_k$, образующее цикл, то выбросим его.

Теперь дерево ещё более оптимальное, чего быть не может.

Какие алгоритмы хорошие (по времени их работы) и почему? Полиномиальные лучше экспоненциальных. Что значит «алгоритм долго работает»?

Для решения задачи можно привлечь разные ресурсы. Если $n^k$ — время, обычно решаем уравнение $n^k \leqslant C$, следовательно, $n \leqslant \sqrt[k]{C}$ — растущая функция.

А что имеем при $ 2^n \leqslant C $. Это хуже!

Оказывается, что почти все задачи (дискретные) в некотором смысле эквивалентны.

То есть умея решить с полиномиальной сложностью задачу №1, могу решить с полиномиальной сложностью задачу №2.

Язык, состоящий из КНФ.

КНФ называется \emph{выполнимой}, если существует набор переменных, на которых КНФ = 1.

МТ: определить, является ??? КНФ легко.

\emph{NP-полный язык} (недетерминированной полиномиальности) Недетерминированная МТ может содержать 2 разных команды
$ q_1 $ и $ q_2 $ в $ L $ и
$ q_1 $ и $ q_2 $ в $ R $.

\begin{enumerate}
\item $ L \in NP $
\item $ L' \in NP, L' \propto L $ (полиномиально сводится)
\end{enumerate}
$ L' \in NP $ эт.б. нек. МТ ??? 1

\begin{theor}
(Кук, Карп, Левин)
\end{theor}

\s{[15.11.11] Лекция 10}

Пусть у нас есть задача распознавания, то есть есть некоторый алфавит $A$, множество слов в этом алфавите и подмножество этого множестве (язык) $L\in A^*$.

Докажем, что $L\in P\ \Leftrightarrow\ \E$ ВМТ, распозн. $L$, и время разбора $\leq P(n)$

$L\in NP\ \Leftrightarrow\ \E$ НМТ

/ Задача на перебор /

Говорят, что один язык полиномиально сводится к другому языку: $L_1\L L_2\colon\E$ ДМТ, работающая за полиномиальное от длины входа врeмя, $x\in L_1\LRA\phi(x)\in L_2$

Утверждение 1: $L\in NP,\ L'\L L \Rightarrow  L'\in NP$

Утверждение 2 (транзитивность полиномиального сведения): $L_1\L L_2, L_2\L L_3 \Rightarrow  L_1\L L_3$

Эталонная задача на перебор -- задача выполнимости

\ss {ВЫП}

$A=\{(, ), x, v, 1, 0, \neg\}$

$\w\in$ВЫП, $(x_1\lor\not\in$ВЫП)

$(x_1\lor x_3)(x_2\lor x_3)\in$ВЫП

\begin{opred}
Язык $L$ -- $NP$-полный, если:\begin{enumerate}
\item $L\in NP$
\item $\F L'\in NP\colon L'\L L$
\end{enumerate}
\end{opred}

\begin{theor}
(Теорема Куна)

Язык выполнимости является $NP$-полным. / без доказательства /
\end{theor}

Утверждение: если язык $L$ -- $NP$-полный, $L'$ -- некоторый другой язык и $L\L L' \Rightarrow  L'$ -- $NP$-полный.

Рассмотрим ещё несколько NP-полных задач.

\ss {КЛИКА}

КЛИКА: дан граф $G$ и некоторое число $k$.

$(G,k)\in$КЛИКА тогда и только тогда, когда в $G$ есть полный подграф из $k$ вершин.
{\theor Язык КЛИКА является NP-полным.}
Докажем две вещи:
\begin{enumerate}
	\item КЛИКА$\in$NP
	\item ВЫП $\mathcal{L}$ КЛИКА
\end{enumerate}
\begin{proof}
На вход подаётся $\w$.

$\w\notin$???, $\w$ - не КНФ, тогда  $(:,2)\in$КЛИКА.

$\w$ -- КНФ $(x_1\lor x_2\lor x_3)(x_2\lor\nx_3)(\nx_1\lor\nx_2)$

$\forall$ литералу в КНФ поставим в соответствие вершину. Рёбра проводятся между всеми парами вершин, кроме
\begin{enumerate}
\item тех, которые соответствуют литералам, стоящим в одной скобке
\item тех, которые соответствуют $x_i$ и $\nx_i$
\end{enumerate}
Если КНФ выполнима, то в $\F$ скобке $\E$ литерал, равный 1.

Вершины, соответствующие истинным литералам, взятые по одному из каждой скобки, образуют клику (они не находятся в одной скобке и не являются отрицанием друг друга, так как оба истинны).

Обратно: если есть некая клика, то в неё входят литералы из разных скобок, им ставим в соответствие 1, остальным - 0 и всё ОК.

(] размер клики=$k$, то \begin{enumerate}
\item литералы, соответствующие разным вершинам, стоят в разных скобках и
\item не являются отрицаниями друг друга.
\end{enumerate}
$x_i\in$КЛИКА, тогда $x_i=1$; $\nx_i\in$КЛИКА, тогда $x_i=0\colon f(\tilde{x})=1$, т.к. все скобки = 1)
\end{proof}

\ss {NM}

NM - независимое множество.

$(G,k):$ существует ли в $G$ множество из $k$ вершин, не соединённых рёбрами?

КЛИКА $\mathcal{L}$ NM

$(G,k)\mapsto(\overline{G},n-k)$

$G=\langle V,E\rangle,\ \overline{G}=\langle V,V^2\setminus E\rangle$

$P\subseteq NP$, т.к. $\F$ ДНТ яляется НДТ ($\subseteq$ или $\subset$, неизвестно (!!!))


\ss {2-ВЫП}

$\w\in$2-ВЫП, если $\w$ -- КНФ, в $\F$ скобке ровно 2 литерала.

2-ВЫП$\in P$

$K=(x_1\lor y_1)(x_1\lor y_2)\ldots(x_1\lor y_k)\&(\overline{x}_1\lor z_1)(\overline{x}_1\lor z_2)\ldots(\overline{x}_1\lor z_l)\&K'(x_2,\ldots,x_n)$

Преобразуем формулу следующим образом:

$=(x_1\lor y_1\ldots y_k)(\overline{x_1}\lor z_1\ldots z_l)K'$

$(x_1\lor x_1) \Rightarrow  x_1=1$

$(\nx_1\lor \nx_1) \Rightarrow \nx=1$

Если литералы различны, то эта формула выполнима тогда и только тогда, когда $\ \underset{1\leq i\leq k}{\underset{1\leq j\leq l}{\&}}(y_i\lor z_j)K'=\Phi'$

$\Phi'$ - выполнима, тогда $\begin{matrix}(x_1\lor y_1\ldots y_k)=1\$x_1\lor z_1\ldots z_l)=1\end{matrix}$

$\left. \begin{matrix}
x_1=1 \Rightarrow  z_1=\ldots=z_l=1\\
x_1=0 \Rightarrow  y_1=\ldots=y_l=1
\end{matrix}\right| \Rightarrow  \Phi'=1$

Обратно: если $\Phi'$ -- выполнима, то \begin{enumerate}
\item все $y_i=1$, следовательно, $x_1=0$ или
\item существует $y_i=0$, тогда $\&(y_i\lor z_j)=0\  \Rightarrow  z_j=1\ \forall j\  \Rightarrow x_1=1$
\end{enumerate}

\begin{lemma}
\E\ биномиальное преобразование $\phi\colon k$ вып. тогда и только тогда, когда $\ \phi(k)$ вып.
\end{lemma}
В $\phi(k)$ на одну переменную меньше.

$(x_1\lor x_2x_4)(\nx_2\lor x_3)(x_3\lor\nx_1)(x_4\lor x_1)(x_4\lor x_3)(\nx_4\lor x_2)$

Устраним $x_1$:

$(x_1\lor x_2x_4)(\nx_1\lor x_3)K'$ выполн. тогда и только тогда, когда $\ (x_2\lor x_3)(x_4\lor x_3)(\nx_2\lor x_4)(x_4\lor x_3)(\nx_4\lor x_2)$ 

Устраним $x_2$:

$(x_2\lor x_3\nx_4)(\nx_2\lor x_3)(x_3\lor x_4)\ \Leftrightarrow\ (x_3\lor x_3)(\nx_4\lor x_3)(x_3\lor x_4),\ x_3=1$

Эта КНФ выполнима, следовательно, исходная КНФ выполнима

\ss {Язык 3-ВЫП}

$\w\in$3-ВЫП тогда и только тогда, когда $\w$ -- КНФ, любая скобка которой содержит ровно 3 литерала.

Задача распознавания 3-выполнимости является NP-полной.

\end{document}

\chapter{Комплексные числа}
\section{Основные понятия}
\begin{opred}
Комплексными числами называются упорядоченые пары вещественных чисел $(a,b),\ a,b\in\R$.
\end{opred}
\mybold{Правила:}
\begin{enumerate}
\item $(a,b)=(c,d)\Leftrightarrow\left\{\begin{matrix}&a=c\\&b=d\end{matrix}\right.$
\item $(a,b)+(c,d)=(a+c,b+d)$
\item $(a,b)(c,d)=(ac-bd,bc+ad)$
\item $(a,0)\equiv a\in\R$
\end{enumerate}
\mybold{Следствия:}
\begin{enumerate}
\item Вычитание: $(a,b)-(c,d)=(a-b,c-d)$
\item Деление: $\cfrac{(a,b)}{(c,d)}=\cfrac{(a,b)(c,-d)}{(c,d)(c,-d)}=\cfrac{(ac+bd,bc-ad)}{(c^2+d^2,0)}$
\end{enumerate}
\mybold{Замечание.}

$(c,-d)$ называется комплексным сопряженным к $(c,d)$.

Коммутативность, ассоциативность, дистрибутивность вытекают из свойств $\R$.

\mybold{Замечание.}

$(0,0)=0,\ (1,0)=1$
\section{Алгебраическая форма записи комплексных чисел}
$(0,1)^2=(0,1)(0,1)=(-1,0)$.

$i=(0,1)$ -- мнимая единица, $i^2=-1$.

$(a,b)=a(1,0)+b(0,1)=a1+bi=a+bi$ -- алгебраическая форма записи комплексного числа.

$z\in\mathbb{C}=(a,b)=a+bi$, $(a,-b)=\overline{z}$

$a=Re\ z$ -- вещественная часть, $b=Im\ z$ -- мнимая часть.
\mybold{Свойства:}
\begin{enumerate}
\item $\overline{\overline{z}}=z$
\item $z\overline{z}=(a^2+b^2,0)=|z^2|;\ |z|=\sqrt{a^2+b^2}$ -- модуль комплексного числа
\item $\overline{z}=z\ \Leftrightarrow\ Im\ z=0$
\item $Re\ z=\frac{z+\overline{z}}2,\ Im\ z=\frac{z-\overline{z}}2$
\item $\overline{z_1\pm z_2}=\overline{z}_1\pm\overline{z}_2$
\end{enumerate}
\section{Комплексная плоскость}
Тут не будет картинки
\section{Сопряженная матрица}
\cmatrixof{A}{a_{kl}}{m}{n}

\cmatrixof{A^*}{\overline{a}_{kl}}{m}{n} -- сопряженная к $A$ матрица.
\section{Тригонометрическая  форма записи комплексных чисел}
$r=|z|=\sqrt{a^2+b^2}$

$a=Re\ z=r\cos\phi$

$b=Im\ z=r\sin\phi$

$\phi=arg\ z$ -- аргумент $z$

$z_1=z_2\ \Leftrightarrow\ \left\{\begin{matrix}|z_1|=|z_2|\\arg\ z_1=arg\ z_2+2\pi k,\ k\in\mathbb{Z}\end{matrix}\right.$

$z=r(\cos\phi+i\sin\phi)$ -- тригонометрическая форма записи комплексного числа

\begin{stat}
$||z_1|-|z_2||\leq|z_1+z_2|\leq|z_1|+|z_2|$
\end{stat}
$z_1z_2=r_1(\cos\phi_1+i\sin\phi_1)r_2(\cos\phi_2+i\sin\phi_2)=r_1r_2(\cos\phi_1\cos\phi_2-\sin\phi_1\sin\phi_2+i(\sin\phi_1\cos\phi_2+\sin\phi_2\cos\phi_1))=$

$=r_1r_2(\cos(\phi_1+\phi_2)+i\sin(\phi_1+\phi_2))$
$|z_1z_2|=|z_1||z_2|$
\begin{effect}
$z^n=r^n(\cos n\phi+i\sin n\phi)$ -- формула Муавра
\end{effect}
$n=0$ \then $z\neq0$
\section{Решение уравнений $z^n=a$ при натуральных $n$}
\begin{opred}
Решение уравнения $z^n=a$ называется корнем $n$-й степени из $a$ ($z,a\in\C$)
\end{opred}
$a=0$ \then $|a|=0$ \then $|z|=0$ \then $z=0$, других корней нет.

Пусть теперь $a\neq0$.
\begin{stat}
У уравнения существует ровно $n$ попарно различных корней.
\end{stat}
\begin{proof}
$\phantom{xxx}$

$z=r(\cos\phi+i\sin\phi)$

$z^n=r^n(\cos(n\phi)+i\sin(n\phi))=a=\rho(\cos\T+i\sin\T)$

Сравним модули: $r^n=\rho$ \then $r=\rho^\frac1n=\sqrt[n]\rho$

Сравниваем три части:

$n\phi=\T+2\pi k,\ k\in\Z$

$\phi=\cfrac{\T+2\pi k}n$

$\phi_k=\cfrac\T n+\cfrac{2\pi k}n,\ k=\overline{0,n-1}$
\end{proof}
\section{Стурктура корней $n$-й степени из 1}
$z^n=a$

Пусть $z_0$ - любой корень этого уравнения. Тогда для $\eps=z/z_0$ получим:

$z=z_0\eps,\ z^n=z_0^n\eps^n=a\eps^n=a$ \then $\eps^n=1$

Следовательно, $z_k=z_0\eps_k$, где $\eps_k^n=1$.

$\eps_k^n=1,\ \eps_0=1,\ \eps_1=\cos\cfrac{2\pi}n+i\sin\cfrac{2\pi}n,\ \eps_k=(\eps_1)^k\ (mod\ n),\ \eps_k\eps_e=\eps_{k+e}\ (mod\ n)$
\chapter{Линейные пространства над произвольным полем}
\section{Основные понятия}
\begin{opred}
Полем называется состоящее из не менее чем двух элементов множество с введенными на нем двумя операциями -- ``сложением'' и ``умножением'' -- обладающими следующими свойствами:
\begin{enumerate}
\item $a+b=b+a$
\item $(a+b)+c=a+(b+c)$
\item $\exists\ 0\colon a+0=a$
\item $\forall\ a\ \exists\ (-a)\colon a+(-a)=0$
\item $ab=ba$
\item $(ab)c=a(bc)$
\item $\exists\ 1\colon 1a=a$
\item $\forall\ a\neq0\ \exists\ \cfrac1a=a\cfrac1a=1$
\item $(a+b)c=ac+bc$
\end{enumerate}
\end{opred}
\begin{remark}
Элементы поля называются числами.
\end{remark}
\begin{opred}
Пусть заданы множество $\V$ и поле $\P$. Множество $\V$ называется линейным (векторным) пространством над полем $\P$, если в $\V$ определены две операции: сложение двух элементов в $\V$ (внутренний закон композиции: $\V\times\V\mapsto\V$) и умножение элементов $\V$ на элементы $\P$ (внешний хакон композиции: $\V\times\P\mapsto\V$), удовлетворяющие следующим аксиомам: $\forall\ a,b,c\in\V,\ \forall\ \alpha,\beta\in\P:$
\begin{enumerate}
\item $a+b=b+a$
\item $(a+b)+c=a+(b+c)$
\item $\exists\ \T\colon a+\T=\T$
\item $\forall\ a\ \exists\ (-a)\colon a+(-a)=\T$
\item $\a(\b a)=(\a\b)a$
\item $1a=a$
\item $(\a+\b)a=\a a+\b a$
\item $\a(a+b)=\a a+\a b$
\end{enumerate}
Элементы линейного пространства называются векторами.
\end{opred}
\begin{opred}
Векторы называются коллинеарными, если они различаются лишь числовыми множителями.
\end{opred}
\section{Линейная зависимость. Ранг и база системы векторов.}
Пусть \V - линейное пространство над полем \P. Будем рассматривать конечные (т.е. состоящие из конечного числа векторов) системы векторов из \V.
\begin{opred}Линейная зависимость\end{opred}
\begin{opred}
Базой системы векторовназывается ее линейно независимая подсистема, через которую линейно выражается любой вектор системы.
\end{opred}
\begin{theor}
Любая линейно независимая подсистема данной системы является ее базой, и, наоборот, всякая база является максимальной линейно независимой подсистемой.
\end{theor}
\begin{proof}
Пусть $a_1,\ldots,a_k$ - система векторов, а $a_1,\ldots,a_r$ - кк максимальная линейно независимая подсистема, $r\leq k$. Тогда необходимо доказать, что любой вектор $a_j(j=\overline{1,k})$ линейно выражается через $a_1,\ldots,a_k$. Если $j\leq r$, то Капитан Очевидность отдыхает. Если $j>r$, то подсистема $a_1,\ldots,a_r,a_j$ линейно зависима \then $\exists\ \a_1,\ldots,\a_r,\a_j$, что $\a_1a_1+\ldots+\a_ra_r+\a_ja_j=\T$. Если $\a_j=0$, то $a_1,\ldots,a_r$ линейно зависима - противоречие \then $\a_j\neq0$ и $a_j=-\cfrac{\a_1}{\a_j}a_1-\ldots-\cfrac{\a_r}{\a_j}a_r$, ч.т.д.

В обратную сторону: необходимо доказать, что база является максимальной линейно независимой подсистемой.

Из определения базы вытекает, что при добавлении к ней любого вектора системы она становится линейно зависимой, так как вновь добавленый вектор выражается через векторы базы.
\end{proof}
\begin{effect}
Все базы данной системы состоят из одинакового числа векторов. Это число есть число векторов в максимальной линейно независимой подсистеме. Оно называется рангом системы: $rg(a_1,\ldots,a_k)$.
\end{effect}
\begin{opred}
Две системы векторов называются эквивалентными, если каждый вектор одной системы линейно выражается через вектора другой системы, и наоборот.
\end{opred}
\begin{effect}
Всякая система эквивалентна своей базе.
\end{effect}
\begin{theor}
Если любой вектор $a_1,\ldots,a_k$ выражается через векторы $b_1,\ldots,b_m$, то $rg(a_1,\ldots,a_k)\leq rg(b_1,\ldots,b_m)$
\end{theor}
\begin{proof}
Заменим системы их базисами и воспользуемся соответствующей теоремой из предыдущего семестра.
\end{proof}
\begin{effect}
\begin{enumerate}
\item Ранги эквивалентных систем совпадают.
\item Базы эквивалентных систем состоят из одинакового числа векторов.
\end{enumerate}
\end{effect}
\section{Базис и его размерность}
\begin{opred}
Говорят, что система векторов (не обязательно линейно независимая) порождает линейное пространство \V, если любой вектор из \V представим в виде линейной комбинации векторов этой системы.
\end{opred}
\begin{opred}
Упорядоченая система векторов называется базисом, если она линейно независима и порождает пространство \V.
\end{opred}
\begin{theor}
Любые два базиса состоят из одинакового числа векторов.
\end{theor}
\begin{proof}
Это утверждение вытекает из эквивалентности двух базисов и следствия 1 теоремы 2.
\end{proof}
\begin{opred}
Количество векторов в базисе называется размерностью пространства \V: $dim\ \V$. Если оно конечно, то ространство называется конечномерным, иначе - бесконечномерным.
\end{opred}
\begin{theor}
В $n$-мерном пространстве любую линейно независимую систему из $k$ векторов ($0\leq k<n$) можно дополнить до базиса.
\end{theor}
\begin{proof}
Пусть векторы $e_1,\ldots,e_k$ построены, $k<n$. Выберем вектор $e_{k+1}$ так, чтобы векторы $e_1,\ldots,e_k,e_{k+1}$ были лнейно независимы. И так далее до получения требуемого результата.
\end{proof}
Свойтсва:

$x=x_1e_1+\ldots+x_ne_n$

$y=y_1e_1+\ldots+y_ne_n$

$\a x+\b y=(\a x_1+\b y_1)e_1+\ldots+(\a x_n+\b y_n)e_n$

Переход к другому базису:

$e=(e_1,\ldots,e_n)$ -- строка из векторов

$x_e=\left(\begin{matrix} x_1 \\ \vdots \\ x_n \end{matrix}\right)$ -- вектор-столбец координат разложения $x$ по базису $e$.

$x=ex_e=fx_f=eQx_f\ \then\ x_e=Qx_f$

Перейдем к другому базису: $f=eQ,\ Q$ -- матрица перехода.
\section{Линейные подпространства. Линейная оболочка.}
\begin{opred}
Подмножество $\L$ линейного пространства $\V$ называется поддпространством (линейного пространства \V), если оно само является линейным пространством относительно операций, введеных в \V.
\end{opred}
\begin{opred}
Линейной оболочкой векторов $a_1,\ldots,a_k$ называется множество всевозможных линейных комбинаций этих векторов. Обозначение: $\lo{\ds a1k}$. В этосм случае говорят, что линейная оболочка натянута на векторы $\ds a1k$.
\end{opred}
\begin{stat}
$\lo{\ds a1k}$ -- подпространство пространства \V.
\end{stat}
\begin{theor}
Две системы векторов эквивалентны тогда и только тогда, когда их линейные оболочки совпадают.
\end{theor}
\begin{proof}
Пусть $a$ и $b$  - две эквивалентные системы, тогда $\lo{b}\subset\lo{a}$ и $\lo{a}\subset\lo{b}$ \then $\lo{a}=\lo{b}$.
\end{proof}
\begin{effect}
$\phantom{dick}$
\begin{enumerate}
\item Линейная оболочка системы векторов совпадает с линейной оболочкой базы этой системы.
\item $dim\ \lo{\ds a1k}=rg(\ds a1k)$
\end{enumerate}
\end{effect}
\begin{theor}
Если $\W\subset\V$, то $dim\ \W\leq dim\ \V$, причем если $dim\ \W=dim\ \V$, то $\W=\V$
\end{theor}
\begin{proof}
Предположим противное. Если $dim\ \W>dim\ \V$, то в $\V$ существует болльше линейно независимых векторов, чем его размерность. Противоречие.

Если $dim\ \W=dim\ \V$, то в качестве базиса $\V$ можно взять базис $\W$ \then $\V=\W$.
\end{proof}
\section{Сумма и пересечение линейных подпространств.}
Два способа задания линейного подпространства: $\lo$ и СЛАУ.
\begin{opred}
Суммой подпространств $\ds \L1k$ пространства $\V$ называется множество всевозможных векторов вида $x=\ds x1k$, где $x_1\in \L_1,\ldots,x_k\in \L_k$. 

Обозначение: $\L_1+\ldots+\L_k$, или $\ind\cup{j=1}k\L_j$, или $\ind\sum{j=1}k\L_j$.
\end{opred}
\begin{opred}
Пересечением подпространств $\ds \L1k$ пространства $\V$ называется множество всевозможных векторов, принадлежащих всем этим подпространствам одновременно.

Обозначение: $\L_1\cap\ldots\cap\L_k$, или $\ind\cap{j=1}k\L_j$.
\end{opred}
\begin{theor}
Сумма и пересечение подпространств является подпространством.
\end{theor}
\begin{proof}
Сумма: очевидно, что сложение м умножение не выводят из пространства.

Пересечение: тоже очевидно.
\end{proof}
\begin{remark}
Сумма подпространств есть наименьшее подространство, содержащее эти подпространства.

Пересечение подпространств есть наибольшее подпространство, содержащееся в этих подпространствах.
\end{remark}
\begin{theor}
Сумма подпространств есть линейная оболочка совокупности базисов этих подпространств.
\end{theor}
\begin{proof}
Пусть $e$ - базис $\L_1$, \ldots, $f$ - базис $\L_k$, тогда $\L_1+\ldots+\L_k\in\lo{e,\ldots,f}$ и $\lo{e,\ldots,f}\in\L_1+\ldots+\L_k$
\end{proof}
\begin{effect}
$dim(\L_1+\ldots+\L_k)=rg(e,\ldots,f)$
\end{effect}
\begin{theor}
Для любых двух подпространств $\L_1$, $\L_2$ справедливо равенство: $dim(\L_1+\L_2)=dim\ \L_1+dim\ \L_2-dim(\L_1\cap\L_2)$
\end{theor}
\begin{proof}
\end{proof}
